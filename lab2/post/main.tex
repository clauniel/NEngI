\documentclass[a4paper]{article}
\usepackage[utf8]{inputenc} % Skal passe til editorens indstillinger
\usepackage[english]{babel} % danske overskrifter


\newcommand{\name}{Carsten Nielsen}
%\newcommand{\stnumber}{s123369, s123161, s123821}
\newcommand{\course}{INI 404 Neuromorphic Engineering~I}
\newcommand{\university}{University of Zürich}
\newcommand{\studyline}{Institute of Neuroinformatics}
\newcommand{\assignment}{Lab 8 Post-Lab}
\renewcommand{\date}{\today} %If another date, than that of today is desiered


% Palatino for rm and math | Helvetica for ss | Courier for tt
\usepackage{mathpazo} % math & rm
\linespread{1.05}        % Palatino needs more leading (space between lines)
\usepackage{palatino} % tt
\normalfont
\usepackage[T1]{fontenc}
\usepackage[english]{babel}

\usepackage{graphicx}%allerese hentet % indsættelse af billeder
\usepackage{epstopdf} %Tilfj "--enable-write18" i argumentet for LaTex build. Dette vil konvertere .eps figurer til pdf-format
\graphicspath{{./picture/}} % stivej til bibliotek med figurer
\usepackage{subcaption} %Til gruppering af figurer
\usepackage{amsmath} %matpakke
\usepackage{amsfonts} %
\usepackage{amssymb} %
\usepackage{steinmetz} % flere matematik symboler
\usepackage{polynom} %for displaying polynom division
\usepackage{mathtools} % matematik - understøtter muligheden for at bruge \eqref{}
\usepackage{float}
\usepackage{placeins}
\usepackage{hhline}

%
\usepackage[usenames,dvipsnames]{xcolor}
\usepackage[compact,explicit]{titlesec}% http://ctan.org/pkg/titlesec
%
\usepackage[europeanresistors]{circuitikz}
\usepackage{pgfplots}
\usepgfplotslibrary{patchplots}
\pgfplotsset{compat=1.11}

%---------%
%Easy edit%
%---------%

%Section formating. arg1 is supplied when making section
\newcommand\presectionnumber[1]{~~}
\newcommand\postsectionnumber[1]{}
\newcommand\midlesection[1]{#1}
\newcommand\sectionnum[1]{\arabic{#1}}
\newcommand\subsectionnum[1]{\arabic{#1}}
\newcommand\subsubsectionnum[1]{\alph{#1}}



%------------%
%setion setup%
%------------%
\renewcommand\thesection{Opgave~\sectionnum{section}} %pas p�, kun i matematik
\renewcommand\thesubsection{\thesection,~\subsectionnum{subsection}}
\definecolor{MagRed}{RGB}{190,40,15}
\definecolor{MathGreen}{RGB}{82,164,0}

\titleformat{\section}{\normalfont\sffamily\large\bfseries\color{MathGreen}}{}{0pt}{|\kern-0.15ex|\kern-0.15ex|\kern-0.15ex|\presectionnumber{#1}\sectionnum{section}\postsectionnumber{#1}\qquad\quad\midlesection{#1}\label{sec:\sectionnum{section}}}
\titleformat{\subsection}[runin]{\large\bfseries}{}{10pt}{\sectionnum{section}.\subsectionnum{subsection})~#1\label{sec:\sectionnum{section}.\subsectionnum{subsection}}}
\titleformat{\subsubsection}[runin]{\itshape}{}{0pt}{\subsectionnum{subsection},\subsubsectionnum{subsection}~#1\label{sec:\sectionnum{section}.\subsectionnum{subsection}.\subsubsectionnum{subsubsection}}}
%\titleformat{\subsubsection}{\bfseries}{}{0pt}{\alph{subsection}.\arabic{subsubsection})\qquad\quad#1\label{\arabic{section}\alph{subsection}\arabic{subsubsection}}}

%----------%
%page setup%
%----------%
\textwidth = 400pt
\marginparwidth = 86pt
\hoffset = -25pt
\voffset= -30pt
\textheight = 670pt

%--------%
%hyperref%
%--------%
\newcommand{\HRule}{\rule{\linewidth}{0.5mm}}
\usepackage{fancyhdr}
\usepackage[plainpages=false,pdfpagelabels,pageanchor=false]{hyperref} % aktive links
\hypersetup{%
  pdfauthor={\name},
  pdftitle={\assignment},
  pdfsubject={\course} }
%\usepackage{memhfixc}% rettelser til hyperref

%-------------%
%Headder setup%
%-------------%
\fancyhf{} % tom header/footer
\fancyhfoffset{20pt}
\fancyhfoffset{20pt}
\fancyhead[OL]{\name \\ INI 404}
\fancyhead[OC]{Date \\ \date}
\fancyhead[OR]{\university\\ \studyline}
\fancyfoot[FL]{}
\fancyfoot[FC]{\thepage}
\fancyfoot[FR]{}
\renewcommand{\headrulewidth}{0.4pt}
\renewcommand{\footrulewidth}{0.4pt}
\headsep = 35pt
\pagestyle{fancy}
 % style setup

%Listings%
\usepackage{listingsutf8}
\usepackage[framed,numbered]{matlab-prettifier}


%setup listings
\lstset{language=Matlab,
  extendedchars=true,
  language=Octave,                % the language of the code
  basicstyle=\ttfamily\footnotesize,           % the size of the fonts that are
  % used for the code
  numbers=left,                   % where to put the line-numbers
  numberstyle=\tiny\color{gray},  % the style that is used for the line-numbers
  stepnumber=2,                   % the step between two line-numbers. If it's 1, each line 
                                  % will be numbered
  numbersep=5pt,                  % how far the line-numbers are from the code
  backgroundcolor=\color{white},      % choose the background color. You must add \usepackage{color}
  showspaces=false,               % show spaces adding particular underscores
  showstringspaces=false,         % underline spaces within strings
  showtabs=false,                 % show tabs within strings adding particular underscores
  frame=single,                   % adds a frame around the code
  rulecolor=\color{black},        % if not set, the frame-color may be changed on line-breaks within not-black text (e.g. comments (green here))
  tabsize=4,                      % sets default tabsize to 2 spaces
  captionpos=b,                   % sets the caption-position to bottom
  breaklines=true,                % sets automatic line breaking
  breakatwhitespace=false,        % sets if automatic breaks should only happen at whitespace
  title=\lstname,                   % show the filename of files included with \lstinputlisting;
                                  % also try caption instead of title
  %keywordstyle=\color{blue},          % keyword style
  %commentstyle=\color{dkgreen},       % comment style
  %stringstyle=\color{mauve},         % string literal style
  escapeinside={\%*}{*)},            % if you want to add LaTeX within your code
  morekeywords={*,...},              % if you want to add more keywords to the set
  deletekeywords={...}              % if you want to delete keywords from the given language
}
\lstset{literate=
  {á}{{\'a}}1 {é}{{\'e}}1 {í}{{\'i}}1 {ó}{{\'o}}1 {ú}{{\'u}}1
  {Á}{{\'A}}1 {É}{{\'E}}1 {Í}{{\'I}}1 {Ó}{{\'O}}1 {Ú}{{\'U}}1
  {à}{{\`a}}1 {è}{{\`e}}1 {ì}{{\`i}}1 {ò}{{\`o}}1 {ù}{{\`u}}1
  {À}{{\`A}}1 {È}{{\'E}}1 {Ì}{{\`I}}1 {Ò}{{\`O}}1 {Ù}{{\`U}}1
  {ä}{{\"a}}1 {ë}{{\"e}}1 {ï}{{\"i}}1 {ö}{{\"o}}1 {ü}{{\"u}}1
  {Ä}{{\"A}}1 {Ë}{{\"E}}1 {Ï}{{\"I}}1 {Ö}{{\"O}}1 {Ü}{{\"U}}1
  {â}{{\^a}}1 {ê}{{\^e}}1 {î}{{\^i}}1 {ô}{{\^o}}1 {û}{{\^u}}1
  {Â}{{\^A}}1 {Ê}{{\^E}}1 {Î}{{\^I}}1 {Ô}{{\^O}}1 {Û}{{\^U}}1
  {œ}{{\oe}}1 {Œ}{{\OE}}1 {æ}{{\ae}}1 {Æ}{{\AE}}1 {ß}{{\ss}}1
  {ç}{{\c c}}1 {Ç}{{\c C}}1 {ø}{{\o}}1 {å}{{\r a}}1 {Å}{{\r A}}1
  {€}{{\EUR}}1 {£}{{\pounds}}1
}

 \lstloadlanguages{% Check Dokumentation for further languages ...
         %[Visual]Basic
         %Pascal
         %C
         %C++
         %XML
         %HTML
         %Java
         %VHDL
         Matlab
 }
 %Listings slut%









%Matematik hurtige ting
%fed
\renewcommand\vec[1]{\mathbf{#1}}
\newcommand\matr[3]{{}_{#2}\mathbf{#1}{}_{#3}}
\newcommand\facit[1]{\underline{\underline{#1}}}
%\renewcommand\d[3]{\frac{\mbox{d}^{#3}#1(#2)}{\mbox{d}#2^{#3}}}
%underline
%\renewcommand\vec[1]{\underline{#1}}
%\newcommand\matr[3]{{}_{#2}\underline{\underline{#1}}{}_{#3}}

\renewcommand\matrix[4]{ %{alignment}{to space}{from space}{matrix}
{\vphantom{\left[\begin{array}{#1}#4\end{array}\right]}}_{#2}\kern-0.5ex
\left[\begin{array}{#1}
#4
\end{array}\right]_{#3}
}
\newcommand\e[0]{\mbox{e}}
\newcommand\E[1]{\cdot 10^{#1}}
\newcommand\im[0]{i}

\newcommand\Jaco{\mbox{Jacobi}}
\newcommand\del[2]{\frac{\partial {#1}}{\partial {#2}}}
\newcommand\abs[1]{\left| {#1} \right|}
\newcommand\stdfig[4]{ %width,img,cap,lab
\begin{figure}[H]
\centering
\includegraphics[width={#1}\textwidth]{#2}
\caption{#3}
\label{#4}
\end{figure}
}
\newcommand\stdfignoscale[3]{ %img,cap,lab
\begin{figure}[H]
\centering
\includegraphics{#1}
\caption{#2}
\label{#3}
\end{figure}
}
\newcommand\diff{\dot}
\newcommand\ddiff{\ddot}
\newcommand\dddiff{\dddot}
\newcommand\ddddiff{\ddddot}






% How to make ref to books or urls in bib
%\citetitle[fx: page 1]{name of ref in bib}

\begin{document}
\begin{titlepage}
\centering \parindent=0pt

\vspace*{\stretch{1}} \HRule\\[1cm]\Huge
\course\\[0.7cm]
\large \assignment\\[1cm]
\HRule\\[4cm]  
%\includegraphics[width=6cm]{picture}\\ Use this if you want a picture on the frontpage
\name\\
%\stnumber
TAs: Ning Quiao, Chenghan Li

\vspace*{\stretch{2}} \normalsize %

\begin{center}
	\date 
\end{center}
\vspace*{\stretch{2}} \normalsize
\begin{flushright}
%\includegraphics[width=6cm]{./dtu.eps}\\
\end{flushright}
\end{titlepage}

\newpage
\section{Post-Lab}
\subsection{How does the thickness of the depletion region depend on the doping and on the
channel (surface) potential? Assume that the doping density is uniform.}

The depth of a depletion layer as a function of the majority and minority carrier concentrations
is given by 
\begin{equation*}
    d = \sqrt{\frac{2\varepsilon_s}{q}\frac{N_A+N_D}{N_AN_D}\psi_s}
\end{equation*}

Even in subthreshold there will be a very small inversion layer under the oxide. This layer will have
a large concetration of holes, for the pMOS, compared to the concentration of the acceptors in the well.

Since
\begin{equation*}
    \frac{N_A+N_D}{N_AN_D} \simeq \frac{1}{N_A} \quad \text{for} \quad N_D \gg N_A
\end{equation*}

The depth of the depletion region can be approximated as

\begin{equation*}
    d = \sqrt{\frac{2\varepsilon_s}{qN_A}\psi_s}
\end{equation*}

\subsection{ Explain why \(\kappa\) varies with the source voltage at constant current (as in the source
follower).}

Because a constant current is flowing through the transistor, the source voltage \(V_s\) and surface potential
\(\psi_s\) are linked such that a fixed amount of electrons diffuse through the channel. When \(V_s\) changes, \(\psi_s\)
must also change in response to keep the current constant. Since we are in the subthreshold region, this change in 
\(\psi_s\) will change the depth of the depletion layer under the oxide. As the depletion layer changes, its capacitance
will also change which will increase or decrease the value of \(\kappa\).

When \(V_s\) increases, \(\psi_s\) will increase and therefore \(\kappa\) will increase because \(C_{dep}\) decreases.
When \(V_s\) decreases, \(\kappa\) likewise decreases.

\subsection{Do the differences in bulk doping account for differences in \(\kappa\) between the native and well devices?}
The well is doped less heavily than the p-substrate. Therefore an increase in \(V_g\) is going to cause larger increase
in the depletion area in a well device than in a native device. A larger increase in the depletion area means that the
capacitance of the depletion layer in the well device is lower, which leads to a higher \(\kappa\) value.

This is indeed what we observed in both experiment 1 and 2, where the pMOS transistor had a higher \(\kappa\) than the 
native nMOS transistor.

\subsection{From your results in Experiments 1 and 2, state under what conditions the assumption that \(\kappa\) 
is constant is reasonable}
In the subthreshold regime, it is reasonable to assume that \(\kappa\) is constant because the changes in its value
are very small as observed in experiment 2. The fit of the linear region in experiment one which represents \(\kappa\) 
over \(U_T\) is also a very good approximation with small residuals, which indicates that \(\kappa\) is approximately 
constant, assuming \(U_T\) is constant.

\subsection{By fabricating a well device in its own well, we can use the well as a second gate -this gate is called a back gate. 
Derive an expression for the current in a saturated well device that shows explicitly the role of the back gate (i.e.,\(V_w\)).}

The equation for \(I_{ds}\) for an nMOS transistor in subthreshold is 
\begin{equation*}
    I_{ds} = I_0e^{\frac{\kappa V_g}{U_T}}\left(e^{\frac{-V_s}{U_T}}-e^{\frac{-V_d}{U_T}}\right)
\end{equation*}

With all variables referenced to the substrate voltage, ground. For the pMOS, the equation is almost the same, the
only difference being that all voltages are measured relative to the voltage of the well \(V_w\). 
The difference is mobility of electrons and holes are contained in \(I_0\). Adding \(V_w\) as the reference to the
above equation we obtain

\begin{equation*}
    I_{ds} = I_0e^{\frac{\kappa(V_w - V_g)}{U_T}}\left(e^{\frac{-(V_w -V_s)}{U_T}}-e^{\frac{-(V_w - V_d)}{U_T}}\right)
\end{equation*}

As the general expression for the current through a pMOS transistor in an n-well, in subthreshold.

\subsection{Design an ideal source follower (ideal in the sense of having unity gain).}
In the source follower, one of the transistors acts as a constant current source. In this case this is the transistor
with the gate input voltage \(V_b\). The problem with the native source follower is that, as we saw in experiment 2,
\(V_s\) and \(V_g\) are related by \(\kappa\). This means that the source follower depicted in the exercise manual 
does not have unity gain. The actual gain is \(\kappa\).

Using two pMOS well type transistors in the same configuration as the nMOS source follower and  with their wells tied to their source,
we can compute \(V_{out}\) as a function of \(V_{in}\), using the equation from problem 5:

\begin{equation*}
    I_{ds} = I_0e^{\frac{\kappa(V_w - V_g)}{U_T}}\left(e^{\frac{-(V_w -V_s)}{U_T}}-e^{\frac{-(V_w - V_d)}{U_T}}\right)
\end{equation*}

The current through the two transistors must always be equal, therefore
\begin{equation*}
    I_0e^{\frac{\kappa(V_{out} - V_{b})}{U_T}}\left(e^{\frac{-(V_{out} -V_{out})}{U_T}}-e^{\frac{-(V_{out} - 0)}{U_T}}\right) = I_0e^{\frac{\kappa(V_{dd} - V_{in})}{U_T}}\left(e^{\frac{-(V_{dd}-V_{dd})}{U_T}}-e^{\frac{-(V_{dd} - V_{out})}{U_T}}\right)
\end{equation*}
Both transistors are in saturation, so we ignore the miniscule backward flowing current from the drain and simplify
\begin{align*}
    e^{\frac{\kappa(V_{out} - V_{b})}{U_T}} &= e^{\frac{\kappa(V_{dd} - V_{in})}{U_T}} \\
    V_{out} = -V_{in} +V_{dd} + V_{b}
\end{align*}
This shows that the circuit has a gain of \(-1\), we therefore swap the \(V_b\) and \(V_{in}\) inputs in the circuit
and obtain
\begin{equation*}
    V_{out} = V_{in} +V_{dd} - V_{b}
\end{equation*}
which has a gain of 1 as required.
\end{document}
