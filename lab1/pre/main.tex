\input{pre}
\begin{document}
\input{frontpage}
\newpage
\section{Prelab}
\subsection{Voltmeters and ammeters.}

An ideal voltmeter as infinite impedance and is placed in parallel with the component to
be measured.

An ideal ammeter has zero impedance and is placed in series with the component to be measured.

\subsection{Draw the test stup you would use to measure the current as a function of voltage through a 
resistor and a diode.}

\begin{figure}
    \center
    \begin{circuitikz}[scale=1.2, american voltages, american resistors]\draw
        (0,0) node {}
        to[V] (0,2.5)
        to[ammeter] (2,2.5)
        to[R] (2,1)
        to[D] (2,0)
        to[short] (0,0)
        (1,0) node[ground] {}
    ;\end{circuitikz}
    \caption{Circuit measuring the current through the resistor and diode as a function of the
    voltage provided by the voltage source.}
    \label{fig:res-diode}
\end{figure}
Fig.~\ref{fig:res-diode} shows a circuit that will measure the current through the resistor and diode
as a function of the voltage provided by the voltage source.

\begin{figure}
    \center
    \begin{subfigure}{0.8\textwidth}
        \center
        \includegraphics{prelab-1-1-2-1.eps}
        \caption{}
    \end{subfigure}

    \begin{subfigure}{0.8\textwidth}
        \center
        \includegraphics{prelab-1-1-2-2.eps}
        \caption{}
    \end{subfigure}
    \caption{Current through the resistor when \textbf{S} is replaced with (a) a voltage source and (b)
a current source.}
    \label{fig:qualia}
\end{figure}

\subsection{Measuring Resistance with a Source-Measure Unit.}
The current through the resistor in Fig.~1.2 of the lab manual for a resistance approaching
infinity can be seen in Fig.~\ref{fig:qualia} when replacing \textbf{S} with (a) a voltage source
and (b) a current source.

If the SMU were to attempt to supply a constant current to an open circuit with infinite resistance,
the voltage would keep rising until it hit the 1100 V limit.

\section{Accuracy of the Keithley SMU}
When using the SMU to calculate resistance, we use Ohm's law \(R = \frac{V}{I}\) and plug in our measurements.
The maximum error will occur when \(I\) is minimal and \(V\) is maximal. In I/V mode supplying 50 \(\mu\)A to
a 100k resistor, the maximum error can be calculated as follows:

The SMU is set to supply a current of 50 \(\mu\)A, the error tolerance at this current is found in the datasheet
to be (0.05\%+20 nA) placing the maximum real current supplied to the resistor at
\begin{equation*}
    I_{real}=I_{meas}(1+0.05\%)+20 nA
\end{equation*}
Where \(I_{meas}\) denotes the "measured" current on the source side.

This will lead to a real voltage across the resistor \(V_{real}=I_{real}R\). Since the resistor is an ideal 100 k\(\Omega\)
resistor, this voltage will be close to 5 V. The measurement error tolerance at 5 V is given in the datasheet as
(0.025\%+1 mV). Since the measurement error is relative to the measured value, and not the real value, the maximum
measured value is given by
\begin{align*}
    V_{real} &= V_{meas}(1-0.025\%)-1mV \\
    &\Updownarrow \\
    V_{meas} &= \frac{V_{real}+1mV}{1-0.025\%}
\end{align*}
For a 100 k\(\Omega\) resistor the maximum measurement error leads to a calculated resistance of 100135 \(\Omega\), an error
of 0.135\%.

In V/I mode the error is calculated in the reverse way, taking into account the change in error tolerances when switching
between source and measure mode, by assuming that the 5 V is the maximum value and that the error is negative on
the source side and positive on the measurement side. This leads to a maximum calculated resistance of 100128 \(\Omega\), an error of 0.128\%.

The same result could have been obtained by converting the absolute error term in the SMU datasheet to a relative error
assuming operating points of 50 \(\mu\)A and 5 V, and combining the relative errors on the source and measurement sides.
This approach works both in V/I and I/V mode.
\end{document}
