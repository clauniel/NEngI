\input{pre}
\tikzset{rrail/.style={rground,yscale=-1}}
\begin{document}
\input{frontpage}
\newpage
\section{Pre-Lab}
\subsection{Write the most general expression for \(I_{ds}\) above threshold in terms of \(V_g,V_s,V_d\)}
Assuming \(\kappa=1\) and that \(V_{Tn} > 0\) and \(V_{Tp} < 0\), we have for the nMOS native transistor:
\begin{equation*}
    I_{ds} = \frac{\beta}{2}\left(\left(V_g-V_{T0}-V_s\right)^2-\left(V_g-V_{T0} - V_d\right)^2\right)
\end{equation*}
And for the well type pMOS transistor:
\begin{equation*}
    I_{ds} = \frac{\beta}{2}\left(\left(-V_g+V_{T0}+V_s\right)^2-\left(-V_g+V_{T0} + V_d\right)^2\right)
\end{equation*}
Where the signs are flipped because the bulk of the pMOS is tied to \(V_dd\) and not ground.

\subsection{Sketch graphs of the drain current vs. the drain Voltage \(V_d\)}
\begin{figure}
    \center
    \includegraphics{prelab2.eps}
    \caption{Qualitative graphs of \(I_{ds}\) vs. \(V_{ds}\) in the above threshold region for varying \(V_g\).
    The ohmic and saturation regions are marked.}
    \label{fig:q2}
\end{figure}
The graphs of the drain currents for varying \(V_{ds}\) and \(V_g\) can be seen in Fig.~\ref{fig:q2}. These
differ from the equivalent graphs in subthreshold mode in that the saturation value of \(V_{ds}\) is not 
constant, but instead depends on the gate overdrive.

\subsection{Derive an expression for the current in the ohmic region, in terms of \(V_g\) and \(V_{ds}\)}
The general equation:
\begin{equation*}
    I_{ds} = \frac{\beta}{2}\left(\left(V_g-V_{T0}-V_s\right)^2-\left(V_g-V_{T0} - V_d\right)^2\right)
\end{equation*}
Can be reduced to 
\begin{equation*}
    I_{ds} = \beta\left(\left(V_g-V_{T0}\right)\left(V_d-V_s\right) - \frac{1}{2}\left(V_d-V_s\right)^2\right)
\end{equation*}
Replacing \(V_d-V_s = V_{ds}\) and assuming that \(V_{ds}\) is very small so the second term can be ignored we
have
\begin{equation*}
    I_{ds} = \beta V_{ds}\left(V_g-V_{T0}\right) 
\end{equation*}
A sketch of the current in the ohmic region can be seen in Fig.~\ref{fig:q3}. The current is proportional to
\(\beta\) and \(V_{ds}\).
\begin{figure}
    \center
    \includegraphics{prelab3.eps}
    \caption{\(I_{ds}\) versus \(V_g\) in the ohmic region.}
    \label{fig:q3}
\end{figure}

\subsection{State the drain voltage condition for above-threshold saturation and derive an expression for the saturation current, \(I_{dsat}\).}

The condition for saturation in the above threshold regime is \(V_{ds} \geq V_g - V_{T0}\).
When we are in saturation, there will be a pinchoff point under the channel with no electrons, therefore the
general equation reduces to 
\begin{equation*}
    I_{dsat} = \frac{\beta}{2}\left(V_g-V_{T0}\right)^2
\end{equation*}
Assuming \(\kappa = 1\). If the y-axis label in graph in Fig.~\ref{fig:q3} is replaced with \(\sqrt{I_{dsat}}\) and the slope replaced with a constant 
1, it shows the saturation current as a function of \(V_g\).

\subsection{Calculate \(C_{ox}\) for the classchip from the values given.}

The capacitance of a capacitor is 
\begin{equation*}
    C=\varepsilon\frac{A}{d}
\end{equation*}
When using the symbols used in the lab manual this becomes

\begin{equation*}
    C_{ox} = \varepsilon_{ox}\varepsilon_0\frac{WL}{t_{ox}}
\end{equation*}
We want the capacitance per square micron, so we calculate
\begin{equation*}
    \frac{C_{ox}}{WL} = \varepsilon_{ox}\varepsilon_0\frac{1}{t_{ox}} = 3.9\cdot8.86\cdot10^{-12}\frac{F}{m}\frac{1}{300\cdot10^{-10}m} = 0.0011518 \frac{F}{m^2} = 1.1518 \frac{fF}{\mu m^2}
\end{equation*}

\subsection{Write the expression for the drain current in saturation including the Early effect.}
The expression for the saturation current when taking the Early effect into account is
\begin{equation*}
    I = I_{dsat}+\left(1+\frac{V_{ds}}{V_E}\right)
\end{equation*}

\subsection{Sketch the setups you will use.}
\begin{figure}
    \begin{subfigure}{0.5\textwidth}
        \center
        \begin{circuitikz}[american voltages] \draw
            (0,0) node[nmos] (mos) {}
            (mos.base) node[anchor=west] {} -- (1,0) to[short] (1,-1)
            (1,-1) node[ground] {} (1,-1)
            (-1.5,-1) to[V] (-1.5,-0) -- (mos.gate)
            (mos.gate) node[anchor=east] {} 
            (-1.5,-1) node[ground] {} (-1.5,-1)
            (mos.drain) node[anchor=south] {} -- (0, 1) 
            (-2,1) to[V=$V_{d}$] (0,1)
            (-2,0.5) node[ground] {} to (-2,1) 
            (mos.source) node[anchor=north] {} to (0,-2)
            (0,-2) node[ground] {} to (0,-2)
            (-2,-0.5) node[anchor=east] {K230}
            (-0.6, 0.3) node[anchor=east] {K236}
        ;\end{circuitikz}
        \caption{}
    \end{subfigure}
    \begin{subfigure}{0.5\textwidth}
        \center
        \begin{circuitikz}[american voltages] \draw
            (0,0) node[pmos] (mos) {}
            (mos.base) node[anchor=west] {} -- (1,0) to[short] (1,1)
            (1,1) node[rrail] {} (1,1)
            (mos.gate) -- (-1.5,0) to[V] (-1.5,1) node[rrail] {}
            (mos.gate) node[anchor=east] {} 
            (mos.source) node[anchor=south] {} -- (0, 1)
            (0,1) node[rrail] {}
            node[right] {$V_{dd}$}
            (2,1) node[rrail] {} to[short] (2,-2) to[short] (0,-2)
            %(0,-2.0) to[V=$V_d$] (mos.drain)
            (mos.drain) to[V_=$V_d$] (0,-2)
            (-2,0.5) node[anchor=east] {K230}
            (+1.4, -1.5) node[anchor=east] {K236}
        ;\end{circuitikz}
        \caption{}
    \end{subfigure}
    \caption{Experimental setup for both pMOS and nMOS transistors that can be used in all three experiments. The TRIAX+ connector of the K236 is always
    connected to the transistor terminal.}
    \label{fig:setup1}
\end{figure}
Fig.~\ref{fig:setup1} shows the setups that will be used for all three experiments. The K236 is placed between the supply rail and drain terminal
of the pMOS in order to use the same rail for lower noise.

\end{document}
