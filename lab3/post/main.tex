\documentclass[a4paper]{article}
\usepackage[utf8]{inputenc} % Skal passe til editorens indstillinger
\usepackage[english]{babel} % danske overskrifter


\newcommand{\name}{Carsten Nielsen}
%\newcommand{\stnumber}{s123369, s123161, s123821}
\newcommand{\course}{INI 404 Neuromorphic Engineering~I}
\newcommand{\university}{University of Zürich}
\newcommand{\studyline}{Institute of Neuroinformatics}
\newcommand{\assignment}{Lab 8 Post-Lab}
\renewcommand{\date}{\today} %If another date, than that of today is desiered


% Palatino for rm and math | Helvetica for ss | Courier for tt
\usepackage{mathpazo} % math & rm
\linespread{1.05}        % Palatino needs more leading (space between lines)
\usepackage{palatino} % tt
\normalfont
\usepackage[T1]{fontenc}
\usepackage[english]{babel}

\usepackage{graphicx}%allerese hentet % indsættelse af billeder
\usepackage{epstopdf} %Tilfj "--enable-write18" i argumentet for LaTex build. Dette vil konvertere .eps figurer til pdf-format
\graphicspath{{./picture/}} % stivej til bibliotek med figurer
\usepackage{subcaption} %Til gruppering af figurer
\usepackage{amsmath} %matpakke
\usepackage{amsfonts} %
\usepackage{amssymb} %
\usepackage{steinmetz} % flere matematik symboler
\usepackage{polynom} %for displaying polynom division
\usepackage{mathtools} % matematik - understøtter muligheden for at bruge \eqref{}
\usepackage{float}
\usepackage{placeins}
\usepackage{hhline}

%
\usepackage[usenames,dvipsnames]{xcolor}
\usepackage[compact,explicit]{titlesec}% http://ctan.org/pkg/titlesec
%
\usepackage[europeanresistors]{circuitikz}
\usepackage{pgfplots}
\usepgfplotslibrary{patchplots}
\pgfplotsset{compat=1.11}

%---------%
%Easy edit%
%---------%

%Section formating. arg1 is supplied when making section
\newcommand\presectionnumber[1]{~~}
\newcommand\postsectionnumber[1]{}
\newcommand\midlesection[1]{#1}
\newcommand\sectionnum[1]{\arabic{#1}}
\newcommand\subsectionnum[1]{\arabic{#1}}
\newcommand\subsubsectionnum[1]{\alph{#1}}



%------------%
%setion setup%
%------------%
\renewcommand\thesection{Opgave~\sectionnum{section}} %pas p�, kun i matematik
\renewcommand\thesubsection{\thesection,~\subsectionnum{subsection}}
\definecolor{MagRed}{RGB}{190,40,15}
\definecolor{MathGreen}{RGB}{82,164,0}

\titleformat{\section}{\normalfont\sffamily\large\bfseries\color{MathGreen}}{}{0pt}{|\kern-0.15ex|\kern-0.15ex|\kern-0.15ex|\presectionnumber{#1}\sectionnum{section}\postsectionnumber{#1}\qquad\quad\midlesection{#1}\label{sec:\sectionnum{section}}}
\titleformat{\subsection}[runin]{\large\bfseries}{}{10pt}{\sectionnum{section}.\subsectionnum{subsection})~#1\label{sec:\sectionnum{section}.\subsectionnum{subsection}}}
\titleformat{\subsubsection}[runin]{\itshape}{}{0pt}{\subsectionnum{subsection},\subsubsectionnum{subsection}~#1\label{sec:\sectionnum{section}.\subsectionnum{subsection}.\subsubsectionnum{subsubsection}}}
%\titleformat{\subsubsection}{\bfseries}{}{0pt}{\alph{subsection}.\arabic{subsubsection})\qquad\quad#1\label{\arabic{section}\alph{subsection}\arabic{subsubsection}}}

%----------%
%page setup%
%----------%
\textwidth = 400pt
\marginparwidth = 86pt
\hoffset = -25pt
\voffset= -30pt
\textheight = 670pt

%--------%
%hyperref%
%--------%
\newcommand{\HRule}{\rule{\linewidth}{0.5mm}}
\usepackage{fancyhdr}
\usepackage[plainpages=false,pdfpagelabels,pageanchor=false]{hyperref} % aktive links
\hypersetup{%
  pdfauthor={\name},
  pdftitle={\assignment},
  pdfsubject={\course} }
%\usepackage{memhfixc}% rettelser til hyperref

%-------------%
%Headder setup%
%-------------%
\fancyhf{} % tom header/footer
\fancyhfoffset{20pt}
\fancyhfoffset{20pt}
\fancyhead[OL]{\name \\ INI 404}
\fancyhead[OC]{Date \\ \date}
\fancyhead[OR]{\university\\ \studyline}
\fancyfoot[FL]{}
\fancyfoot[FC]{\thepage}
\fancyfoot[FR]{}
\renewcommand{\headrulewidth}{0.4pt}
\renewcommand{\footrulewidth}{0.4pt}
\headsep = 35pt
\pagestyle{fancy}
 % style setup

%Listings%
\usepackage{listingsutf8}
\usepackage[framed,numbered]{matlab-prettifier}


%setup listings
\lstset{language=Matlab,
  extendedchars=true,
  language=Octave,                % the language of the code
  basicstyle=\ttfamily\footnotesize,           % the size of the fonts that are
  % used for the code
  numbers=left,                   % where to put the line-numbers
  numberstyle=\tiny\color{gray},  % the style that is used for the line-numbers
  stepnumber=2,                   % the step between two line-numbers. If it's 1, each line 
                                  % will be numbered
  numbersep=5pt,                  % how far the line-numbers are from the code
  backgroundcolor=\color{white},      % choose the background color. You must add \usepackage{color}
  showspaces=false,               % show spaces adding particular underscores
  showstringspaces=false,         % underline spaces within strings
  showtabs=false,                 % show tabs within strings adding particular underscores
  frame=single,                   % adds a frame around the code
  rulecolor=\color{black},        % if not set, the frame-color may be changed on line-breaks within not-black text (e.g. comments (green here))
  tabsize=4,                      % sets default tabsize to 2 spaces
  captionpos=b,                   % sets the caption-position to bottom
  breaklines=true,                % sets automatic line breaking
  breakatwhitespace=false,        % sets if automatic breaks should only happen at whitespace
  title=\lstname,                   % show the filename of files included with \lstinputlisting;
                                  % also try caption instead of title
  %keywordstyle=\color{blue},          % keyword style
  %commentstyle=\color{dkgreen},       % comment style
  %stringstyle=\color{mauve},         % string literal style
  escapeinside={\%*}{*)},            % if you want to add LaTeX within your code
  morekeywords={*,...},              % if you want to add more keywords to the set
  deletekeywords={...}              % if you want to delete keywords from the given language
}
\lstset{literate=
  {á}{{\'a}}1 {é}{{\'e}}1 {í}{{\'i}}1 {ó}{{\'o}}1 {ú}{{\'u}}1
  {Á}{{\'A}}1 {É}{{\'E}}1 {Í}{{\'I}}1 {Ó}{{\'O}}1 {Ú}{{\'U}}1
  {à}{{\`a}}1 {è}{{\`e}}1 {ì}{{\`i}}1 {ò}{{\`o}}1 {ù}{{\`u}}1
  {À}{{\`A}}1 {È}{{\'E}}1 {Ì}{{\`I}}1 {Ò}{{\`O}}1 {Ù}{{\`U}}1
  {ä}{{\"a}}1 {ë}{{\"e}}1 {ï}{{\"i}}1 {ö}{{\"o}}1 {ü}{{\"u}}1
  {Ä}{{\"A}}1 {Ë}{{\"E}}1 {Ï}{{\"I}}1 {Ö}{{\"O}}1 {Ü}{{\"U}}1
  {â}{{\^a}}1 {ê}{{\^e}}1 {î}{{\^i}}1 {ô}{{\^o}}1 {û}{{\^u}}1
  {Â}{{\^A}}1 {Ê}{{\^E}}1 {Î}{{\^I}}1 {Ô}{{\^O}}1 {Û}{{\^U}}1
  {œ}{{\oe}}1 {Œ}{{\OE}}1 {æ}{{\ae}}1 {Æ}{{\AE}}1 {ß}{{\ss}}1
  {ç}{{\c c}}1 {Ç}{{\c C}}1 {ø}{{\o}}1 {å}{{\r a}}1 {Å}{{\r A}}1
  {€}{{\EUR}}1 {£}{{\pounds}}1
}

 \lstloadlanguages{% Check Dokumentation for further languages ...
         %[Visual]Basic
         %Pascal
         %C
         %C++
         %XML
         %HTML
         %Java
         %VHDL
         Matlab
 }
 %Listings slut%









%Matematik hurtige ting
%fed
\renewcommand\vec[1]{\mathbf{#1}}
\newcommand\matr[3]{{}_{#2}\mathbf{#1}{}_{#3}}
\newcommand\facit[1]{\underline{\underline{#1}}}
%\renewcommand\d[3]{\frac{\mbox{d}^{#3}#1(#2)}{\mbox{d}#2^{#3}}}
%underline
%\renewcommand\vec[1]{\underline{#1}}
%\newcommand\matr[3]{{}_{#2}\underline{\underline{#1}}{}_{#3}}

\renewcommand\matrix[4]{ %{alignment}{to space}{from space}{matrix}
{\vphantom{\left[\begin{array}{#1}#4\end{array}\right]}}_{#2}\kern-0.5ex
\left[\begin{array}{#1}
#4
\end{array}\right]_{#3}
}
\newcommand\e[0]{\mbox{e}}
\newcommand\E[1]{\cdot 10^{#1}}
\newcommand\im[0]{i}

\newcommand\Jaco{\mbox{Jacobi}}
\newcommand\del[2]{\frac{\partial {#1}}{\partial {#2}}}
\newcommand\abs[1]{\left| {#1} \right|}
\newcommand\stdfig[4]{ %width,img,cap,lab
\begin{figure}[H]
\centering
\includegraphics[width={#1}\textwidth]{#2}
\caption{#3}
\label{#4}
\end{figure}
}
\newcommand\stdfignoscale[3]{ %img,cap,lab
\begin{figure}[H]
\centering
\includegraphics{#1}
\caption{#2}
\label{#3}
\end{figure}
}
\newcommand\diff{\dot}
\newcommand\ddiff{\ddot}
\newcommand\dddiff{\dddot}
\newcommand\ddddiff{\ddddot}






% How to make ref to books or urls in bib
%\citetitle[fx: page 1]{name of ref in bib}


\tikzset{rrail/.style={rground,yscale=-1}}
\newcommand{\reffig}[1]{Fig.~\ref{#1}}

\begin{document}
\begin{titlepage}
\centering \parindent=0pt

\vspace*{\stretch{1}} \HRule\\[1cm]\Huge
\course\\[0.7cm]
\large \assignment\\[1cm]
\HRule\\[4cm]  
%\includegraphics[width=6cm]{picture}\\ Use this if you want a picture on the frontpage
\name\\
%\stnumber
TAs: Ning Quiao, Chenghan Li

\vspace*{\stretch{2}} \normalsize %

\begin{center}
	\date 
\end{center}
\vspace*{\stretch{2}} \normalsize
\begin{flushright}
%\includegraphics[width=6cm]{./dtu.eps}\\
\end{flushright}
\end{titlepage}

\newpage
\section{Post-Lab}
\subsection{Give an intuitive explanation to why the relationship between the above-threshold current
and the gate voltage is linear in the ohmic region, but quadratic in the saturation region.}

In the ohmic region, there is both a forward and reverse current flowing in the transistor. The currents
flow by both drift and diffusion and their magnitude depends on the amount of charge available at the drain
and source ends of the channel. The charge available at the source and drain ends of the channel depends
on the applied gate voltage, and the charge available at the drain end also depends on the drain voltage.

The current flow depends on the difference of the squared charges, so while \(V_{ds}\) is small compared to \(V_g\), this is approximately
linear with \(V_{ds}\). Therefore the current in the ohmic region is approximately linear.

In the saturation region, there is no charge at the drain end of the channel because \(V_{ds}\) is so large compared
to \(V_g\) that it will force the charge away from the drain, creating a pinchoff point. Now the difference in
the squared charges is a quadratic function only dependent on \(V_s\).

If \(V_{ds}\) is kept constant in the ohmic region, the amount of charge added to the channel ends increases approximately
linear with \(V_g\) for the same reason as it increases linearly with \(V_{ds}\). An increase in the gate voltage will 
create additional charge at both ends of the channel.

In the saturation regime, additional charge needed because of an increase in \(V_g\) can only appear at the source, and
so the relationship between current and gate voltage must be quadratic because the current depends on the difference
in charge squared between the channel ends.

\subsection{In the subthreshold region: (a) Is the current linearly proportional to the gate voltage in the ohmic region
like it is above threshold? (b) Does the drain-source voltage at which the current saturates depend on the gate voltage?}

(a) In the subthreshold region. The current that flows is mainly a diffusion current which is exponentially dependent on the
amount of charge available at both ends of the channel. In the triode region, the current is technically an exponential 
function of the gate voltage, but because \(V_{ds}\) is small, it can be reduced to an approximately linear function.

(b) Since the current flows mainly by diffusion in the subthreshold regime, the start of the saturation region does not
depends on the gate voltage like in the superthreshold regime. This is simply because there ceases to be a reverse current
diffusing from the drain towards the source, and this lets the diffusion current from the source dominate. The point at which
this begins to happens depends only on \(V_{ds}\) and not \(V_g\), and therefore the start of the saturation region does
not depend on the gate voltage.

\subsection{Compute the current through a diode-connected transistor with source voltage \(V_s\) and drain voltage \(V_d\).
    Explain how it behaves like a diode when connected like this. Draw a diode and show how you will replace it with (a) an n-fet
and (b) a p-fet.}

\begin{figure}[!htb]
    \begin{subfigure}{0.3\textwidth}
        \center
        \begin{circuitikz}\draw
            (0,0) to[D] (1,0)
            (0,0) node[anchor=south east] {\(V_a\)}
            (1,0) node[anchor=south west] {\(V_b\)}
        ;\end{circuitikz}
        \caption{}
    \end{subfigure}
    \begin{subfigure}{0.3\textwidth}
        \center
        \begin{circuitikz}\draw
            (0,0) node[nmos] (mos) {}
            (mos.gate) node[anchor=east] {} -- ++(0,0.5)  to[short, -*] (mos.drain |- {{(0,0.5)}})
            (mos.drain) node[anchor=west] {\(V_d = V_a\)}
            (mos.source) node[anchor=west] {\(V_s = V_b\)}
        ;\end{circuitikz}
        \caption{}
    \end{subfigure}
    \begin{subfigure}{0.3\textwidth}
        \center
        \begin{circuitikz}\draw
            (0,0) node[pmos] (mos) {}
            (mos.gate) node[anchor=east] {} -- ++(0,-0.5)  to[short, -*] (mos.drain |- {{(0,-0.5)}})
            (mos.drain) node[anchor=west] {\(V_d = V_b\)}
            (mos.source) node[anchor=west] {\(V_s = V_a\)}
        ;\end{circuitikz}
        \caption{}
    \end{subfigure}
    \caption{(a) A diode, (b) diode-connected nMOS transistor and (c) diode-connected pMOS transistor. The gate is tied to the drain in both cases.}
    \label{fig:diode}
\end{figure}
In subthreshold, the current through a transistor is given by the following equation if we assume that \(V_s\) is small.
\begin{equation*}
    I = I_0e^{\frac{\kappa V_g - V_s}{U_T}}\left(1-e^{- \frac{V_d-V_s}{U_t}}\right)
\end{equation*}
If we connect the drain and gate terminals this becomes
\begin{equation*}
    I = I_0e^{\frac{\kappa V_g - V_s}{U_T}}\left(1 - e^{-\frac{V_g-V_s}{U_t}}\right)
\end{equation*}
As long as the transistor is operating in saturation, this will be approximately
\begin{equation*}
    I = I_0e^{\frac{\kappa V_g - V_s}{U_T}}
\end{equation*}
And since the transistor is always going to be in saturation because \(V_g\) is connected directly to \(V_d\),
locking the drain into being reverse biased with respect to the channel voltage, this is always true.

The above equation is an exponential which depends only on the voltage between drain and source, the same as for
a diode where the current is an exponential function of the voltage across the diode.

Fig.~\ref{fig:diode} shows a diode with voltage labels and the diode connected nMOS and pMOS transistors with matching 
labels.

\subsection{When a diode transistor is operating forward biased, is it operating in the ohmic region or saturation region? 
Does it making any difference whether it is biased in sub- or superthreshold?}
The transistor is operating in the saturation region as explained previously. It must also be operating in subthreshold to function like
a diode since this is the region where the current is expontentially dependant on the gate voltage (and thus drain voltage).

\subsection{Basic mechanism of amplifier design using MOSFET}
In subthreshold operation the transconductance of the gate and drain are
\begin{align*}
    g_g &= \frac{\partial I}{\partial V_g} = \frac{\kappa I}{U_T} \\
    g_d &= \frac{\partial I}{\partial V_d} = \frac{I}{V_E}
\end{align*}
Where \(V_E\) is the Early voltage. Therefore we have, in subthreshold
\begin{align*}
    A = \frac{\partial V_d}{\partial V_g} = \frac{V_E}{I}\frac{\kappa I}{U_T} = \frac{\kappa V_E }{U_T}
\end{align*}

Above threshold in the saturation region, the conductances above are
\begin{align*}
    g_g &= \frac{\partial I}{\partial V_g} = \beta\left(\kappa\left(V_g - V_T\right) - V_s\right) \\
    g_d &= \frac{\partial I}{\partial V_d} = \frac{I}{V_E}
\end{align*}
Which leads to the following gain
\begin{align*}
    A = \frac{\partial V_d}{\partial V_g} = \frac{V_E\beta\left(\kappa\left(V_g - V_T\right) - V_s\right)}{I} 
\end{align*}
And since the current is \(I_b\) and \(V_s=0\) this becomes
\begin{align*}
    A = \frac{\partial V_d}{\partial V_g} = \frac{V_E\beta\kappa\left(V_g - V_T\right)}{I_b} 
\end{align*}

\end{document}
