\input{pre}

\tikzset{rrail/.style={rground,yscale=-1}}
\newcommand{\reffig}[1]{Fig.~\ref{#1}}

\begin{document}
\input{frontpage}
\newpage
\section{Post Lab}
The FI curve can not be brought to saturation by adjusting the other parameters in the circuit. Unless \(V_\tau\) or the threshold voltage is
adjusted such that the input voltage becomes the limiting factor.

For the Axon-Hillock circuit, the frequency limiting factor is either going to be the input current, the switching speed of the second inverter, or
the discharge rate through the membrane discharge transistors. If the input current is not a limitation, the circuit will either saturate at 
a spike frequency if the switching speed of the inverter is limiting. Otherwise, if the discharge current cannot keep up with the input current,
the output will settle to \(V_{dd}\) and no more spikes will be produces. Of course the circuit will then be destroyed if charge is continously pumped
onto the membrane capacitor without sufficient discharge.

Since \(V_{in}\) is the gate voltage for a transistor operating in sub-threshold, the input current to the circuit will be an exponential
function of \(V_{in}\). Therefore a semilog-plot with \(V_{in}\) on a log scale and the frequency on a linear scale will be equivalent to
a linear FI plot.
\end{document}
