\input{pre}
\tikzset{rrail/.style={rground,yscale=-1}}
\begin{document}
\input{frontpage}
\newpage
\section{Passive Properties and Conductances}
\subsection{Passive Properties}
In general, passive properties are properties which are always present and unchanging. For the neuron, the leakage resistance
and membrane capacitance are considered passive properties. The sodium, potassium and chloride conductances can also be considered
passive for modelling purposes even though they are voltage dependant.

\subsection{Modelling Passive Properties}
With discrete circuit elements, neurons can be modeled using resistors and capacitors. In VLSI, the resistors are replaced with 
transistors.

\subsection{Relevant Conductances in Spike-Generation}
The relevant conductances involved in generating an action potential are the sodium and potassium conductances. First the 
sodium channels are activated, increasing the sodium conductance, then potassium channels open which increases the potassium conductance while
the sodium channels are closed again.

Another important conductance is the chloride conductance which functions as an inhibitory conductance.

\section{Power Dissipation}
\subsection{Concerns for the Axon-Hillock Circuit}
The membrane potential in the axon-hillock circuit spends a lot of time hovering around the threshold voltage. This dissipates a lot of
power in the first inverter of the inverter pair, since both the p-fet and n-fet are halfway on most of the time. Therefore, a large current
will flow through the first inverter without serving a particular function.

The second inverter does not dissipate nearly as much power as the first inverter, since the positive feedback guarantees that as soon as the input
to the second inverter approaches the critical point, it will flip over. Therefore it spends less time in a state where both its transistors
are conducting large currents.

\section{FI-curves and Refractory Period}
\subsection{The Meaning of \emph{Refractory} period}
The refractory period consists of two sub-periods; the \emph{absolute} and \emph{relative} refractory periods. The absolute refractory period
is the period of timer after a spike in which it is impossible to make a neuron fire again. The relative period is the period after a 
spike in which the neuron can be made to fire, provided it receives an input larger than it would need if it was at its resting potential.

\subsection{The Cause of Refractory Periods}
Immediately after a neuron spikes, the sodium channels are completely closed and some time must pass before they can be opened again. Therefore
any attempt to depolarize the neuron will fail because the sodium current responsible for depolarization simply cannot flow.

After the absolute period ends, a large influx of potassium will have hyperpolarized the cell to a membrane potential lower than its resting potential.
Therefore, even though the sodium channels can now open, it will take more charge to bring the membrane potential to threshold than it would from the 
resting potential.

\subsection{What is an FI-curve}
The FI-curve is the relationship between the input current to a cell, and that cell's firing frequency. For any cell, there will be a minimum current
required for firing, which must be larger than the leakage current. Also, there is a maximum current, above which any current will not increase the 
firing rate because the absolute refractory period is limiting the neuron.
\end{document}
