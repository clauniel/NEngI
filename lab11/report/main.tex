\input{pre}

\newcommand{\reffig}[1]{Fig.~\ref{#1}}

\begin{document}
\input{frontpage}
\newpage
In this lab we study the behavior of a low-power integrate and fire neuron using the classchip 2005rev2008. First, we observe the effect of changing different parameters and we try to obtain a biologically plausible response. Then, we measure some frequency vs. input current characteristics. \\

\section{Single Spike Plots}

We tuned the parameters to obtain a spiking frequency of around 100 Hz and a refractory period of a few milliseconds. 

We are interested in the voltage that is equivalent to the membrane potential in biological neurons. However, this voltage cannot be measured directly as the circuit is very sensitive to current flowing into or out of that node. Therefore, we measure it indirectly using a transistor whose gate is connected to the node. Depending on the node's voltage, the transistor will produce a current. This current is then transformed back into a voltage using an operational amplifier with a feedback resistor. The voltage produced will thus be related to the "membrane potential" although not linearly. The results can be seen in Fig~\ref{fig:exp1a}. 
\begin{figure}[!h]
    \center
    \includegraphics{exp1a.eps}
    \caption{Indirect measurement of the "membrane potential" of a spiking neuron circuit for different values of the voltage that controls the refractory period, $V_{ref}$. The rest of the parameters remain constant at $V_{in}=4.150 V$, $V_{thr}=0.706 V$ and $V_{tau}=0.282 V$. }
    \label{fig:exp1a}
\end{figure}

\section{F-I Curves}
Now we measure the spike frequency of the circuit as a function of the input current. Here again, we rely on indirect measurements as we know the voltage applied to the gate of the transistor that produces the input current instead of the current itself. The results are shown in Fig~\ref{fig:exp1b} for different values of $V_{ref}$.

As expected, the higher the value of $V_{ref}$, the higher the saturation frequency. This is because the refractory time where the neuron is irresponsive diminishes. \\

\begin{figure}[!h]
	\center
	\includegraphics{exp1b.eps}
	\caption{Spike frequency of the circuit in logarithmic scale as a function of the input voltage. This is qualitatively equivalent to plotting the frequency in natural units against the input current as the input current depends exponentialy on the voltage.}
	\label{fig:exp1b}
\end{figure}


\section{Spike Frequency Adaptation}


\end{document}
