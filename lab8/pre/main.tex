\input{pre}
\tikzset{rrail/.style={rground,yscale=-1}}
\begin{document}
\input{frontpage}
\newpage
\section{Pre-Lab}
\subsection{P-N junction diagram}
See appended figure.
\subsection{Generation of hole pairs}
When a foton generates an electron-hole pair in the N-region, the electron will appear in the conduction band and
the hole in the valence band. The electron will either move around in the N region for some time, or immediately recombine
with the hole that was generated or some other hole. The hole will either recombine with an electron, or diffuse into
the depletion region in which case it will be swept across to the P-region by the electric field in the depletion region.

If the carrier pair is generated in the P-region, the same opportunities exist as for the case in the N-region, only 
now the electron is the carrier that can cross the depletion region.

If the pair is generated in the depletion region, the electron will be push into the N-region and the hole will be pushed
into the P-region, by the electric field.

\subsection{Shining light on an open junction}
If we shine light on an open-circuted junction, the process described in the previous question will cause a reverse current
to flow across the junction. Since no net current can actually flow because it is not a closed circuit, the buildup of
charge caused by the reverse current, increases the voltage across the junction and begins to drive a forward current
equal and opposite to the reverse current. The sign of the generated voltage is positive when measured from P to N.

\subsection{Shining light on a shorted junction}
If the junction is shorted, current can flow. When we shine light on the junction, the reverse current will therefore 
not be balanced by an equal forward current, but flow from N to P.

\subsection{I-V relationship of the photodiode}
See appended figure for a graph of the I-V relationship with the maximum power operating point. 

The optimal power point can also be analytically derived. The current through the photodiode is given by
\begin{equation*}
    I = I_0\left(e^{V}{U_T}-1\right)-I_{ph}
\end{equation*}
Where \(I_{ph}\) is the reverse current caused by phototransduction. The power is then
\begin{equation*}
    P = VI = VI_0\left(e^{\frac{V}{U_T}}-1\right) -VI_{ph}
\end{equation*}
The optimal power point is then
\begin{equation*}
    \frac{dP}{dV} = I_0e^{\frac{V}{U_T}}\left(V+1\right) - I_{ph} - I_0 = 0
\end{equation*}
Or more clearly
\begin{equation*}
    I_0e^{\frac{V}{U_T}}\left(\frac{V}{U_T}+1\right) = I_{ph} + I_0
\end{equation*}

\subsection{Usefulness of logarithmic sensors in perception}
A logarithmic sensor is able to operate over a wide range of inputs because the logarithm is a compressive operation.
Additionally, the logarithmic response allows one to use, in the case of photosensors, the ambient lighting as an 
operating point. The result of small changes in ligthing due to objects will be encoded as small changes around this operating
point, which allows for the creation of a small signal model.

Logarithms also have the property that a ratio between two variables is turned into a difference, i.e.
\begin{equation*}
    \log\left(\frac{x}{y}\right) = \log(x) - \log(y)
\end{equation*}
Which allows one to easily work with ratios in a circuit.

\subsection{The source-follower photoreceptor}
The output voltage of the source-follower photorecepter has the following V-I relationship
\begin{equation*}
    V_{out} = \kappa V_b - U_T\log\left(\frac{I_{ph}}{I_0}\right)
\end{equation*}
Or when rewriting \(I_{ph}\) in terms of irradiance
\begin{equation*}
    V_{out} = \kappa V_b - U_T\log\left(\frac{\eta q\lambda P_{opt}}{I_0hc}\right)
\end{equation*}
The slope of the response is
\begin{equation*}
    \frac{dV_{out}}{dI_{ph}} = \frac{U_T}{I_{ph}}
\end{equation*}

\end{document}
