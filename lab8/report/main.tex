\input{pre}

\tikzset{rrail/.style={rground,yscale=-1}}
\newcommand{\reffig}[1]{Fig.~\ref{#1}}

\begin{document}
\input{frontpage}
\newpage
\begin{figure}
    \center
    \includegraphics{srcfoll.eps}
    \caption{Output voltage as a function of relative light intensity for the source-follower receptor. Strips 1 through 4
    are decade light intensity filters, numbered in descending opacity. The dark current line is the maximum output voltage since
the always-present dark current fixes the voltage to this value.}
    \label{fig:srcfoll}
\end{figure}
\section{Quantum Efficiency of a Photodiode}
Using the photometer we measured the following irradiances for the red and green LEDs
\begin{equation*}
    E_{red} = 2.52 \frac{\mathrm{W}}{\mathrm{m}^2} \; ,\; E_{green} = 2.22 \frac{\mathrm{W}}{\mathrm{m}^2}
\end{equation*}
When applying a reverse bias across the diode of 5 V and shielding it from as much light as possible, the dark
current is approximately
\begin{equation*}
    I_0 = 0.0049 \mathrm{nA}
\end{equation*}
The reverse current through the diode when shining light on the junction using the LEDs at maximum intensity is
\begin{equation*}
    I_{red} = 0.4523 \mathrm{nA} \; , \; I_{green} = 0.6282 \mathrm{nA}
\end{equation*}
We find the quantum efficiency as
\begin{equation*}
    \eta = \frac{hc I_{ph}}{q\lambda P_{opt}}
\end{equation*}
Where \(P_{opt}\) is the incoming optical power to the junction and \(I_{ph} = I-I_0\). \(P_{opt}\) can be calculated using the area of the junction
which is given to be 98 square microns and the irradiance of the light source. 

The quantum efficiencies for both light sources are
\begin{equation*}
    \eta_{red} = 3.53 \; , \; \eta_{green} = 6.29
\end{equation*}
These numbers are obviously not correct since the quantum efficiency is always lower than 1. Also, the quantum efficiency for red light
is lower than that of green light, which does not immediately make sense given that the absorbtion spectrum of silicon peaks at a wavelength of 800nm.
Therefore the red light should have the highest quantum efficiency as photons with this wavelength have enough energy to generate electron-hole
pairs, but not so much energy that they are immediately absorbed in the surfce of the n-region, and not so little energy that they cannot 
generate electron-hole pairs or pass through the material entirely.

However, the junction depth can modulate the efficiency for red and green light. A junction nearer to the surface of the substrate will be more
efficient for green light compared to red light, as most of the higher energy green light will be absorbed near the surface (in the junction) and
most of the red light will penetrate deeper into the substrate and therefore not contribute appreciably to the photocurrent.

\section{Source-Follower Receptor}
We biased the source follower bias transistor at 1 V and observed that the output went close to the ground rail when shining light on the chip,
while it hovered just below 0.7 V when we blocked as much light as possible from reaching the chip.

Fig.~\ref{fig:srcfoll} shows the response of the source-follower receptor for different light intensities, relative to the maximum intensity
we were able to produce using the green LED with no filter in place.

Blocking as much light as possible so only the dark current remains results in the maximum possible output voltage. The output voltage can never
go higher than this value, as long as the bias does not change, because of the dark current. Therefore the response of the circuit will saturate
and this voltage for low intensities and cease to be an actual log response.

For the higher intensities, the bias transistor goes above threshold and therefore the output voltage drops very quickly with increasing light intensity
because the I-V relationship is no longer exponential, but quadratic. This is also shown on the figure as shining LED light on the chip without any filter
very clearly causes the output voltage to drop to almost 0.


\end{document}
