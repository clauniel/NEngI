\input{pre}
\tikzset{rrail/.style={rground,yscale=-1}}
\begin{document}
\input{frontpage}
\newpage
\section{Floating Nodes in MOS Technology}
The following calculation shows the effectiveness of \(\mathrm{SiO}_2\) as an insulator. The leakage current is computed as
\begin{equation*}
    I_{leak} = I_0e^{\frac{-3.2 \mathrm{eV}}{kT}}
\end{equation*}
Where \(I_0 = qNAv\). With the values given in the exercise manual, the leakage current is approximately
\begin{equation*}
    I_{leak} = 1.13 \cdot 10^{-51} \mathrm{A}
\end{equation*}
Which is an electron flow of 1 electron per \(10^{32}\) seconds.

\section{Operational Amplifier}
\subsection{The output voltage follows the input voltage}
When the input to the opamp changes, the difference between the two signals will be amplified and applied to the output.
Because the output is connected to the negative input terminal, the difference between the positive and negative input terminal
will immediately be brought to zero.

Analytically:
\begin{equation*}
    V_{out} = A(V_{in}^+ - V_{in}^-)
\end{equation*}
But because of the negative feedback connection
\begin{align*}
    V_{out} &= A(V_{in}^+ - V_{out}) \\
            &\Updownarrow \\
    V_{out} &= V_{in}^+ \frac{A}{A+1}
\end{align*}
For large values of \(A\)
\begin{equation*}
    V_{out} \simeq V_{in}
\end{equation*}

\subsection{Operating Curves}
Figs here.

\section{Lab Circuit}
See Fig for the updated circuit.
\subsection{Hot electron injection}
Hot electron occurs (HEI) when high-energy holes move into the pFET transistor's drain and knock free an electron. With a high
field strength between the drain and channel of the transistor, the electrons will accelerate rapidly and some may cross through the
oxide and onto the floating gate. Others may become trapped in the oxide and damage it.

\subsection{Electrons added at constant rate using HEI}
See Fig

\subsection{Tunneling injection}
Tunneling can work in two ways, either the oxide is thin enough that the electrons can tunnel directly through. This is called direct tunneling.
The other option is Fowler-Nordheim tunneling, which needs a well process. Here the well is put at a very high potential compared to the floating
gate such that the energy level of the oxide conduction band is distorted across space. This lowers the effective distance that electrons have to
tunnel through in order to make it through the oxide.

\subsection{Removing electrons using tunneling}
See Fig.

\subsection{Obtaining current from voltage vs. time graphs}
We know that the voltage changes because charge is removed from the floating gate node which is effectively one side of a capacitor plate. 
Therefore the current is simply the derivative of the voltage with respect to time, multiplied by the size of the capacitor.
\begin{equation*}
    I = C\frac{dV}{dt}
\end{equation*}

\subsection{Assumption of constant current}
It is reasonable to assume a constant current because the voltages used to force the HEI and tunneling effects are very large compared to the
floating gate node voltage. Otherwise the current would not be constant, but instead resemble the charging of a capacitor to a specific voltage.

\section{Application of a High voltage to a CMOS Gate}
See Fig.

\end{document}
