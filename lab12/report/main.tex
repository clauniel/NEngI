\input{pre}

\tikzset{rrail/.style={rground,yscale=-1}}
\newcommand{\reffig}[1]{Fig.~\ref{#1}}

\begin{document}
\input{frontpage}
\newpage

In this lab we test a non-volatile analog memory cell. The cell stores charge in an isolated 'floating' gate using quantum tunneling and hot electron injection. 

The chips that were  available to us did not seem to work properly so we are using the data obtained by the group of M. Milde et al. for this report. \\

\section{Measuring Injection and Tunneling Current}

Here we show examples of injection and tunneling and demonstrate how to calculate the currents that flow into the floating gate. 

In Fig.~\ref{fig:exp1a} we can see how the output voltage of the circuit increases linearly as we inject electrons into the floating gate. The injection current can be calculated as follows:

\begin{equation*}
	V_{out}\approx V_{inp}+V_c
\end{equation*}

\begin{equation*}
	\frac{dV_{out}}{dt}\approx \frac{dV_c}{dt} = \frac{I_{inj}}{C}
\end{equation*}
So that
\begin{equation*}
	I_{inj} = C\frac{dV_{out}}{dt}
\end{equation*}
In the case of Fig.~\ref{fig:exp1a}, the injection current will therefore be 
\begin{equation*}
	I_{inj}=2\cdot10^{-12}\cdot0.0714 = 1.43\cdot10^{-13} A\\
\end{equation*}

\begin{figure}[!h]
	\center
	\includegraphics{exp1a.eps}
	\caption{Output voltage of the circuit as we inject electrons into the floating gate with $V_{inj}=5 V$. The line corresponds to a linear fit.}
	\label{fig:exp1a}
\end{figure}

Fig.~\ref{fig:exp1b} shows the output of the circuit as we perform tunneling. Similarly, the tunneling current can be calculated as

\begin{equation*}
	I_{tun} = -C\frac{dV_{out}}{dt}
\end{equation*}

So we obtain 
\begin{equation*}
	I_{inj}=-2\cdot10^{-12}\cdot(-0.1682) = 3.36\cdot10^{-13} A\\
\end{equation*}

\begin{figure}[!h]
	\center
	\includegraphics{exp1b.eps}
	\caption{Output voltage of the circuit as we tunnel electrons out of the floating gate and a linear fit.}
	\label{fig:exp1b}
\end{figure}

\section{pFET Injection}

In this experiment we measure the output voltage of the circuit for five different values of $V_{inj}$ between 10 V and 10.5 V and infer the injection currents as explained above. 

The efficiency of the injection (injection current divided by the current through the injection transistor) is plotted in Fig.~\ref{fig:exp2}.

\begin{figure}[!h]
	\center
	\includegraphics{exp2.eps}
	\caption{Injection efficiency ($I_{inj}/I_{transistor}$) as a function of the source current, $I_{transistor}$. The line represents the linear fit: $Efficiency = 2.6886\cdot10^{-5} I_{transistor} -2\cdot10^{-9}$.}
	\label{fig:exp2}
\end{figure}

The injection current can be characterized by 
\begin{equation*}
	I_{inj} = I_{transistor}~e^{\frac{V_{dc}}{V_{i}}}
\end{equation*}
where $V_i$ is a constant and $V_{dc}$ is the drain to channel voltage in the injection transistor. In this term, the drain voltage is constant (0 V) and the channel voltage is approximately constant. This is so because the channel voltage is related to the floating gate voltage, and this is kept close to $V_{inp}$ by the transconductance amplifier and the negative feedback. Therefore, it makes sense that the injection current is approximately proportional to the current through the transistor as we can see in Fig.~\ref{fig:exp2}.
\\
\section{Gate Oxide Tunneling}

Lastly, we compute the tunneling currents for five different values of $V_{tun}$ between 27 and 32 V. Figs.~\ref{fig:exp3a} and \ref{fig:exp3b} show the results. 
\begin{figure}[!h]
	\center
	\includegraphics{exp3a.eps}
	\caption{Tunneling current as a function of the oxide voltage, $V_{ox}$ (i.e. $V_{tun}-V_{fg}$). $V_{fg}$ is assumed to be equal to $V_{inp}$. The curve represents a fit with a model of the form $I_{tun}=I_0e^{-\frac{V_0}{V_{ox}}}$.}
	\label{fig:exp3a}
\end{figure}


\begin{figure}[!h]
	\center
	\includegraphics{exp3b.eps}
	\caption{Tunneling current in logarithmic scale, $log(I_{tun})$, as a function of $-1/V_{ox}$ and a linear fit (solid line). The dashed line corresponds to the exponential fit of Fig.~\ref{fig:exp3a}.}
	\label{fig:exp3b}
\end{figure}

The data has been fitted with a model of the form $I_{tun}=I_0e^{-\frac{V_0}{V_{ox}}}$. In this model, $I_0$ indicates value at which the tunneling current saturates and $V_{0}$ is a constant that depends on the thickness of the oxide. The thicker the oxide, the higher the oxide voltage that needs to be applied in order to get a certain current. 

In Fig.~\ref{fig:exp3a} the fit is done using the model directly and the original data. We obtain $I_0=0.22235$ A and $V_{0}=752.83$ V. In Fig.~\ref{fig:exp3b}, a line is fitted to the logarithm of the current as a function of $-1/V_{ox}$. With this we obtain $I_0=85.129$ A and $V_{0}=908.54$ V.

The model fits the data reasonably well. However, the points measured lie in the initial part of the equation that resembles a positive exponential, so we do not get to see the change in curvature and saturation of the current that the model predicts for higher voltages. 


\end{document}
