\input{pre}

\tikzset{rrail/.style={rground,yscale=-1}}
\newcommand{\reffig}[1]{Fig.~\ref{#1}}

\begin{document}
\input{frontpage}
\newpage
\section{Post-Lab}
\subsection{Zero output current at non-zero input voltage with fixed output voltage.}
Assuming that the question pertains to the simple transconductance amplifier, this effect
is caused by the output voltage affecting the operating point on the saturation curve of the
output transistor. Because of the Early voltage, the output transistor will be able to handle
slightly more or less current than the non-output transistor when its drain voltage is fixed.

This also means that the non-zero input voltage, at which zero output current is observed, will
change with the fixed output voltage.
\subsection{nMOS and pMOS simple transconductance amplifiers}
\begin{figure}
    \begin{subfigure}{0.5\textwidth}
        \center
        \begin{circuitikz}[american voltages, american resistors]\draw
            (0,0) node[nmos] (Mb) {}
            (Mb.source) to[short, i=$I_b$] ++(0,-0.1) node[ground] {}
            (Mb.gate) node[anchor=east] {$V_{b}$}
            (-1,2) node[nmos] (M1) {}
            (M1.source) to[short, i=$I_1$] (M1.source |- Mb.drain) to[short] (Mb.drain)
            (1,2) node[nmos, xscale=-1] (M2) {}
            (M2.source) to[short, i=$I_2$] (M2.source |- Mb.drain) to[short, -*] (Mb.drain)
            (-1,4) node[pmos, xscale=-1] (M3) {}
            (1,4) node[pmos] (M4) {}
            (M3.drain) to[short] (M1.drain)
            (M3.drain) to[short, *-] (M3.drain -| M3.gate) to[short, -*] (M3.gate)
            (M4.drain) to[short, -*] (M2.drain)
            (M4.source) node[rrail] {}
            (M3.source) node[rrail] {}
            (M1.gate) node[anchor=south] {$V_1+$}
            (M2.gate) node[anchor=south] {$V_2-$}
            (M2.drain) to[short, i=$I_{out}$] ++(1,0) node[anchor=west] {$V_{out}$}
        ;\end{circuitikz}
        \caption{}
    \end{subfigure}
    \begin{subfigure}{0.5\textwidth}
        \center
        \begin{circuitikz}[american voltages, american resistors]\draw
            (0,0) node[pmos] (Mb) {}
            (Mb.source) to[short, i<=$I_b$] ++(0,+0.1) node[rrail] {}
            (Mb.gate) node[anchor=east] {$V_{b}$}
            (-1,-2) node[pmos] (M1) {}
            (M1.source) to[short, i<=$I_1$] (M1.source |- Mb.drain) to[short] (Mb.drain)
            (1,-2) node[pmos, xscale=-1] (M2) {}
            (M2.source) to[short, i<=$I_2$] (M2.source |- Mb.drain) to[short, -*] (Mb.drain)
            (-1,-4) node[nmos, xscale=-1] (M3) {}
            (1,-4) node[nmos] (M4) {}
            (M3.drain) to[short] (M1.drain)
            (M3.drain) to[short, *-] (M3.drain -| M3.gate) to[short, -*] (M3.gate)
            (M4.drain) to[short, -*] (M2.drain)
            (M4.source) node[ground] {}
            (M3.source) node[ground] {}
            (M1.gate) node[anchor=south] {$V_1-$}
            (M2.gate) node[anchor=south] {$V_2+$}
            (M2.drain) to[short, i=$I_{out}$] ++(1,0) node[anchor=west] {$V_{out}$}
        ;\end{circuitikz}
        \caption{}
    \end{subfigure}
    \caption{The simple transconductance amplifier built using nMOS (a) and pMOS (b) transistors for the differential pair. 
        The inverting and non-inverting inputs are flipped, and the voltages input bias voltages must now be negative and referenced
    to \(V_{dd}\).} 
    \label{fig:q2}
\end{figure}
Fig.~\ref{fig:q2} shows the simple transconductance amplifier constructed using nMOS and pMOS transistors.

The conditions for keeping \(M_b\) in saturation are the same for the two transconductance amplifiers, except that
the signs of the input and bias voltages are positive and negative, and referenced to ground and \(V_{dd}\) for nMOS and
pMOS configurations respectively.
\subsection{Advantages of the wide-range vs. simple transconductance amplifier.}
The wide-range transconductance amplifier uses more transistors than the simple one, 9 vs. 5, and therefore consumes almost double the
area in the layout. Because of transistor mismatch, the current mirrors used in both amplifiers cannot copy currents precisely and this
effect is worsened in the wide-range amplifier because it has a series of 3 current mirrors.

On the other hand, the wide-range transconductance amplifier can handle a higher dynamic range of inputs, and when configured in the open
circuit configuration it does not behave like a source follower for negative differential voltages before applying a large gain.
\end{document}
