\input{pre}
\tikzset{rrail/.style={rground,yscale=-1}}
\begin{document}
\input{frontpage}
\newpage
\section{Pre-Lab}
\subsection{2-Node WTA Network}
\begin{figure}
    \center
    \begin{circuitikz}\draw
        %WTA 1
        (-1,0) node[anchor=south west] (C) {$V_c$}
        (-1,-2) node[nmos] (Mb1) {}
        (-3,0) node[nmos,xscale=-1] (M1) {} 
        (M1) node[anchor=south east] {$M_1$}
        (-1, 2) node[nmos] (M2) {}
        (M2) node[anchor=south west] {$M_2$}
        (-1,4) node[pmos,xscale=-1] (Mp2) {}
        (-3,4) node[pmos] (Mp1) {}
        (1,4) node[pmos] (Mp3) {}
        %WTA 2
        (6,-2) node[nmos] (Mb2) {}
        (4,0) node[nmos,xscale=-1] (M3) {}
        (M3) node[anchor=south east] {$M_3$}
        (6,0) node[anchor=south west] {$V_c$}
        (4,4) node[pmos] (Mp4) {}
        (6,2) node[nmos] (M4) {}
        (M4) node[anchor=west] {$M_4$}
        (6,4) node[pmos,xscale=-1] (Mp5) {}
        (8,4) node[pmos] (Mp6) {}
        %ground connectections
        (M1.source) node[ground] {}
        (Mb1.source) node[ground] {}
        (Mb2.source) node[ground] {}
        (M3.source) node[ground] {}
        %Vdd connections
        (Mp1.source) node[rrail] {}
        (Mp2.source) node[rrail] {}
        (Mp3.source) node[rrail] {}
        (Mp4.source) node[rrail] {}
        (Mp5.source) node[rrail] {}
        (Mp6.source) node[rrail] {}
        %inputs
        (Mb1.gate) node[anchor=south] {$V_b$}
        (Mb2.gate) node[anchor=south] {$V_b$}
        (Mp1.gate) node[anchor=south] {$V_{i1}$}
        (Mp4.gate) node[anchor=south] {$V_{i_2}$}
        %outputs
        (Mp3.drain) to[short,i=$I_{o1}$] ++(0,-0.1)
        (Mp6.drain) to[short,i=$I_{o2}$] ++(0,-0.1)
        %interconnect
        (Mp2.gate) to[short, *-] (Mp2.gate |- Mp2.drain) to[short,-*] (Mp2.drain)
        (Mp5.gate) to[short, *-] (Mp5.gate |- Mp5.drain) to[short,-*] (Mp5.drain)

        (Mp2.drain) to[short] (M2.drain)
        (Mp5.drain) to[short] (M4.drain)

        (Mp1.drain) to[short] (M1.drain)
        (Mp4.drain) to[short] (M3.drain)
        (M2.gate) to[short, -*] (M1.drain |- M2.gate)
        (M4.gate) to[short, -*] (M3.drain |- M4.gate)

        (M2.source) to[short] (Mb1.drain)
        (M4.source) to[short] (Mb2.drain)

        (M1.gate) to[short,-*] (-1,0) to[short,-*] (6,0)
        %intermediate signals
        (M2.gate) node[anchor=south] {$V_{o1}$}
        (M4.gate) node[anchor=south] {$V_{o2}$}
        (Mp1.drain) to[short,i=$I_{i1}$] ++(0,-0.1)
        (Mp4.drain) to[short,i=$I_{i2}$] ++(0,-0.1)
    ;\end{circuitikz}
    \caption{}
\end{figure}
When \(I_{i1}=I_{i2}\) it follows that \(I_{o1}=I_{o2}\) since for \(M_1\):
\begin{equation*}
    I_{i1} = I_0e^{\frac{\kappa V_c - 0}{U_T}}-I_0e^{\frac{\kappa V_c - V_{o1}}{U_T}}
\end{equation*}
And for \(M_2\)
\begin{equation*}
    I_{i2} = I_0e^{\frac{\kappa V_c - 0}{U_T}}-I_0e^{\frac{\kappa V_c - V_{o2}}{U_T}}
\end{equation*}
Where the second term in both equations is the reverse component and the first term is the forward component.
The only solution to the above equations are that \(V_{o1}=V_{o2}\) and therefore \(I_{o1}=I_{o2}=I_b\), 
meaning both \(M_2\) and \(M_4\) are conducting.

When \(I_{i1} \gg I_{i2}\), \(V_c\) must increase in order to cope with the increased current being forced through \(M_1\), causing
\(M_3\) to open further. However, the current through \(M_3\) is now very low compared to its gate voltage and therefore \(V_{o2}\) drops,
causing \(M_4\) to enter the deep triode region or even cutoff. Therefore 
\begin{align*}
    I_{o2} &\simeq 0 \\
    I_{o1} &\simeq 2I_b \\
    V_{o1} &\gg V_{o2} 
\end{align*}

This can be generalized to the \(n\)-input case. Here, for \(I_{i1}=I_{i2}=\ldots=I_{in}\) the output currents will be
\begin{equation*}
    I_{o1}=I_{o2}=\ldots=I_{on}=\frac{I_b}{n}
\end{equation*}
And for one input current \(I_{ix}\) much larger than any other input, then 
\begin{equation*}
    I_{ox} = nI_b
\end{equation*}

\subsection{Small Signal Model}
Applying a small differential input current, we have
\begin{equation*}
    I_i = I_u \pm \frac{1}{2}\Delta I_{i}
\end{equation*}
Because of the Early effect, the only way that transistors \(M_1\) and \(M_3\) can adapt to the changing current going through them
is to modify their drain voltages, assuming that \(V_c\) is constant (see appendix Fig. 1). The change in drain voltage is given by the drain conductance and
change in input current
\begin{equation*}
    \Delta V_o = \frac{\Delta I_i}{g_d}
\end{equation*}
Assuming positive \(\Delta I\) makes \(I_{i1}\) larger, this leads to
\begin{align*}
    I_{i1} = I_u+\frac{1}{2}\Delta I \quad & \quad I_{i2} = I_u-\frac{1}{2}\Delta I \\
                                           &\Downarrow \\
    V_{o1} = V_u+\frac{1}{2}\frac{\Delta I}{g_d} \quad & \quad V_{o2} = V_u-\frac{1}{2}\frac{\Delta I}{g_d} 
\end{align*}
As we are in subthreshold saturation, 
\begin{equation*}
    g_d = \frac{I_{sat}}{V_E}
\end{equation*}
And when using the bias current \(I_u\)
\begin{align*}
    I_{sat} \simeq \frac{I_u}{1+\frac{V_{ds}}{V_E}}
\end{align*}
However, since \(V_E\) is very large, this can be approximated as
\begin{equation*}
    I_{sat} \simeq I_u
\end{equation*}
And therefore
\begin{equation*}
    g_d = \frac{I_u}{V_E}
\end{equation*}

The change in output current is given by the change in output voltage and the gate transconductance of \(M_2\) and \(M_4\)
\begin{equation*}
    \Delta I_o = g_m\Delta V_o
\end{equation*}
Substituting \(\Delta V_o = \frac{\Delta I_i}{g_d}\) this becomes the input-output current relation is
\begin{equation*}
    \frac{\Delta I_{o}}{\Delta I_i} = \frac{g_m}{g_d}
\end{equation*}
Since each output current is biased at \(I_u=I_b\) the gate transconductance is
\begin{equation*}
    g_m = \frac{\kappa I_b}{U_T}
\end{equation*}
The input-output relation is then
\begin{equation*}
    \frac{\Delta I_o}{\Delta I_i} = \frac{\kappa I_b V_E}{I_u U_T}
\end{equation*}
And when normalized by \(I_b\) and \(I_i\)
\begin{equation*}
    \frac{\Delta I_o I_u}{\Delta I_i I_b} = \frac{\kappa V_E}{U_T}
\end{equation*}
Normal values of \(V_E\) are 150-750 V, \(\kappa\) is between 0.6-1 and \(U_T=26\) mV at room temperature. Using
\(V_E=450\) V and \(\kappa=0.8\) the normalized gain is
\begin{equation*}
    A_I = \frac{\Delta I_o I_u}{\Delta I_i I_b} = \frac{0.8\cdot450 V}{26\cdot 10^{-3} V} \simeq 14\cdot10^{3}
\end{equation*}

The assumption that \(V_c\) does not change holds for small differential currents because if \(V_c\) went up or down it would
try to force both the currents through \(M_1\) and \(M_4\) up or down with it according to the transconductance of the transistors.
This is impossible since we are applying a differential input current which forces one current up and the other down in relation 
to the common bias.

When operating above threshold, \(g_m\) for transistors \(M_2\) and \(M_4\) changes to 
\begin{equation*}
    g_m = \beta\left(\kappa\left(V_o-V_T\right)-V_c\right)
\end{equation*}
While the drain conductance stays the same.

\subsection{Draw the Setups}
See appendix.
\end{document}
