\documentclass[a4paper]{article}
\usepackage[utf8]{inputenc} % Skal passe til editorens indstillinger
\usepackage[english]{babel} % danske overskrifter


\newcommand{\name}{Carsten Nielsen}
%\newcommand{\stnumber}{s123369, s123161, s123821}
\newcommand{\course}{INI 404 Neuromorphic Engineering~I}
\newcommand{\university}{University of Zürich}
\newcommand{\studyline}{Institute of Neuroinformatics}
\newcommand{\assignment}{Lab 8 Post-Lab}
\renewcommand{\date}{\today} %If another date, than that of today is desiered


% Palatino for rm and math | Helvetica for ss | Courier for tt
\usepackage{mathpazo} % math & rm
\linespread{1.05}        % Palatino needs more leading (space between lines)
\usepackage{palatino} % tt
\normalfont
\usepackage[T1]{fontenc}
\usepackage[english]{babel}

\usepackage{graphicx}%allerese hentet % indsættelse af billeder
\usepackage{epstopdf} %Tilfj "--enable-write18" i argumentet for LaTex build. Dette vil konvertere .eps figurer til pdf-format
\graphicspath{{./picture/}} % stivej til bibliotek med figurer
\usepackage{subcaption} %Til gruppering af figurer
\usepackage{amsmath} %matpakke
\usepackage{amsfonts} %
\usepackage{amssymb} %
\usepackage{steinmetz} % flere matematik symboler
\usepackage{polynom} %for displaying polynom division
\usepackage{mathtools} % matematik - understøtter muligheden for at bruge \eqref{}
\usepackage{float}
\usepackage{placeins}
\usepackage{hhline}

%
\usepackage[usenames,dvipsnames]{xcolor}
\usepackage[compact,explicit]{titlesec}% http://ctan.org/pkg/titlesec
%
\usepackage[europeanresistors]{circuitikz}
\usepackage{pgfplots}
\usepgfplotslibrary{patchplots}
\pgfplotsset{compat=1.11}

%---------%
%Easy edit%
%---------%

%Section formating. arg1 is supplied when making section
\newcommand\presectionnumber[1]{~~}
\newcommand\postsectionnumber[1]{}
\newcommand\midlesection[1]{#1}
\newcommand\sectionnum[1]{\arabic{#1}}
\newcommand\subsectionnum[1]{\arabic{#1}}
\newcommand\subsubsectionnum[1]{\alph{#1}}



%------------%
%setion setup%
%------------%
\renewcommand\thesection{Opgave~\sectionnum{section}} %pas p�, kun i matematik
\renewcommand\thesubsection{\thesection,~\subsectionnum{subsection}}
\definecolor{MagRed}{RGB}{190,40,15}
\definecolor{MathGreen}{RGB}{82,164,0}

\titleformat{\section}{\normalfont\sffamily\large\bfseries\color{MathGreen}}{}{0pt}{|\kern-0.15ex|\kern-0.15ex|\kern-0.15ex|\presectionnumber{#1}\sectionnum{section}\postsectionnumber{#1}\qquad\quad\midlesection{#1}\label{sec:\sectionnum{section}}}
\titleformat{\subsection}[runin]{\large\bfseries}{}{10pt}{\sectionnum{section}.\subsectionnum{subsection})~#1\label{sec:\sectionnum{section}.\subsectionnum{subsection}}}
\titleformat{\subsubsection}[runin]{\itshape}{}{0pt}{\subsectionnum{subsection},\subsubsectionnum{subsection}~#1\label{sec:\sectionnum{section}.\subsectionnum{subsection}.\subsubsectionnum{subsubsection}}}
%\titleformat{\subsubsection}{\bfseries}{}{0pt}{\alph{subsection}.\arabic{subsubsection})\qquad\quad#1\label{\arabic{section}\alph{subsection}\arabic{subsubsection}}}

%----------%
%page setup%
%----------%
\textwidth = 400pt
\marginparwidth = 86pt
\hoffset = -25pt
\voffset= -30pt
\textheight = 670pt

%--------%
%hyperref%
%--------%
\newcommand{\HRule}{\rule{\linewidth}{0.5mm}}
\usepackage{fancyhdr}
\usepackage[plainpages=false,pdfpagelabels,pageanchor=false]{hyperref} % aktive links
\hypersetup{%
  pdfauthor={\name},
  pdftitle={\assignment},
  pdfsubject={\course} }
%\usepackage{memhfixc}% rettelser til hyperref

%-------------%
%Headder setup%
%-------------%
\fancyhf{} % tom header/footer
\fancyhfoffset{20pt}
\fancyhfoffset{20pt}
\fancyhead[OL]{\name \\ INI 404}
\fancyhead[OC]{Date \\ \date}
\fancyhead[OR]{\university\\ \studyline}
\fancyfoot[FL]{}
\fancyfoot[FC]{\thepage}
\fancyfoot[FR]{}
\renewcommand{\headrulewidth}{0.4pt}
\renewcommand{\footrulewidth}{0.4pt}
\headsep = 35pt
\pagestyle{fancy}
 % style setup

%Listings%
\usepackage{listingsutf8}
\usepackage[framed,numbered]{matlab-prettifier}


%setup listings
\lstset{language=Matlab,
  extendedchars=true,
  language=Octave,                % the language of the code
  basicstyle=\ttfamily\footnotesize,           % the size of the fonts that are
  % used for the code
  numbers=left,                   % where to put the line-numbers
  numberstyle=\tiny\color{gray},  % the style that is used for the line-numbers
  stepnumber=2,                   % the step between two line-numbers. If it's 1, each line 
                                  % will be numbered
  numbersep=5pt,                  % how far the line-numbers are from the code
  backgroundcolor=\color{white},      % choose the background color. You must add \usepackage{color}
  showspaces=false,               % show spaces adding particular underscores
  showstringspaces=false,         % underline spaces within strings
  showtabs=false,                 % show tabs within strings adding particular underscores
  frame=single,                   % adds a frame around the code
  rulecolor=\color{black},        % if not set, the frame-color may be changed on line-breaks within not-black text (e.g. comments (green here))
  tabsize=4,                      % sets default tabsize to 2 spaces
  captionpos=b,                   % sets the caption-position to bottom
  breaklines=true,                % sets automatic line breaking
  breakatwhitespace=false,        % sets if automatic breaks should only happen at whitespace
  title=\lstname,                   % show the filename of files included with \lstinputlisting;
                                  % also try caption instead of title
  %keywordstyle=\color{blue},          % keyword style
  %commentstyle=\color{dkgreen},       % comment style
  %stringstyle=\color{mauve},         % string literal style
  escapeinside={\%*}{*)},            % if you want to add LaTeX within your code
  morekeywords={*,...},              % if you want to add more keywords to the set
  deletekeywords={...}              % if you want to delete keywords from the given language
}
\lstset{literate=
  {á}{{\'a}}1 {é}{{\'e}}1 {í}{{\'i}}1 {ó}{{\'o}}1 {ú}{{\'u}}1
  {Á}{{\'A}}1 {É}{{\'E}}1 {Í}{{\'I}}1 {Ó}{{\'O}}1 {Ú}{{\'U}}1
  {à}{{\`a}}1 {è}{{\`e}}1 {ì}{{\`i}}1 {ò}{{\`o}}1 {ù}{{\`u}}1
  {À}{{\`A}}1 {È}{{\'E}}1 {Ì}{{\`I}}1 {Ò}{{\`O}}1 {Ù}{{\`U}}1
  {ä}{{\"a}}1 {ë}{{\"e}}1 {ï}{{\"i}}1 {ö}{{\"o}}1 {ü}{{\"u}}1
  {Ä}{{\"A}}1 {Ë}{{\"E}}1 {Ï}{{\"I}}1 {Ö}{{\"O}}1 {Ü}{{\"U}}1
  {â}{{\^a}}1 {ê}{{\^e}}1 {î}{{\^i}}1 {ô}{{\^o}}1 {û}{{\^u}}1
  {Â}{{\^A}}1 {Ê}{{\^E}}1 {Î}{{\^I}}1 {Ô}{{\^O}}1 {Û}{{\^U}}1
  {œ}{{\oe}}1 {Œ}{{\OE}}1 {æ}{{\ae}}1 {Æ}{{\AE}}1 {ß}{{\ss}}1
  {ç}{{\c c}}1 {Ç}{{\c C}}1 {ø}{{\o}}1 {å}{{\r a}}1 {Å}{{\r A}}1
  {€}{{\EUR}}1 {£}{{\pounds}}1
}

 \lstloadlanguages{% Check Dokumentation for further languages ...
         %[Visual]Basic
         %Pascal
         %C
         %C++
         %XML
         %HTML
         %Java
         %VHDL
         Matlab
 }
 %Listings slut%









%Matematik hurtige ting
%fed
\renewcommand\vec[1]{\mathbf{#1}}
\newcommand\matr[3]{{}_{#2}\mathbf{#1}{}_{#3}}
\newcommand\facit[1]{\underline{\underline{#1}}}
%\renewcommand\d[3]{\frac{\mbox{d}^{#3}#1(#2)}{\mbox{d}#2^{#3}}}
%underline
%\renewcommand\vec[1]{\underline{#1}}
%\newcommand\matr[3]{{}_{#2}\underline{\underline{#1}}{}_{#3}}

\renewcommand\matrix[4]{ %{alignment}{to space}{from space}{matrix}
{\vphantom{\left[\begin{array}{#1}#4\end{array}\right]}}_{#2}\kern-0.5ex
\left[\begin{array}{#1}
#4
\end{array}\right]_{#3}
}
\newcommand\e[0]{\mbox{e}}
\newcommand\E[1]{\cdot 10^{#1}}
\newcommand\im[0]{i}

\newcommand\Jaco{\mbox{Jacobi}}
\newcommand\del[2]{\frac{\partial {#1}}{\partial {#2}}}
\newcommand\abs[1]{\left| {#1} \right|}
\newcommand\stdfig[4]{ %width,img,cap,lab
\begin{figure}[H]
\centering
\includegraphics[width={#1}\textwidth]{#2}
\caption{#3}
\label{#4}
\end{figure}
}
\newcommand\stdfignoscale[3]{ %img,cap,lab
\begin{figure}[H]
\centering
\includegraphics{#1}
\caption{#2}
\label{#3}
\end{figure}
}
\newcommand\diff{\dot}
\newcommand\ddiff{\ddot}
\newcommand\dddiff{\dddot}
\newcommand\ddddiff{\ddddot}






% How to make ref to books or urls in bib
%\citetitle[fx: page 1]{name of ref in bib}

\tikzset{rrail/.style={rground,yscale=-1}}
\begin{document}
\begin{titlepage}
\centering \parindent=0pt

\vspace*{\stretch{1}} \HRule\\[1cm]\Huge
\course\\[0.7cm]
\large \assignment\\[1cm]
\HRule\\[4cm]  
%\includegraphics[width=6cm]{picture}\\ Use this if you want a picture on the frontpage
\name\\
%\stnumber
TAs: Ning Quiao, Chenghan Li

\vspace*{\stretch{2}} \normalsize %

\begin{center}
	\date 
\end{center}
\vspace*{\stretch{2}} \normalsize
\begin{flushright}
%\includegraphics[width=6cm]{./dtu.eps}\\
\end{flushright}
\end{titlepage}

\newpage
\section{Pre-Lab}
\subsection{2-Node WTA Network}
\begin{figure}
    \center
    \begin{circuitikz}\draw
        %WTA 1
        (-1,0) node[anchor=south west] (C) {$V_c$}
        (-1,-2) node[nmos] (Mb1) {}
        (-3,0) node[nmos,xscale=-1] (M1) {} 
        (M1) node[anchor=south east] {$M_1$}
        (-1, 2) node[nmos] (M2) {}
        (M2) node[anchor=south west] {$M_2$}
        (-1,4) node[pmos,xscale=-1] (Mp2) {}
        (-3,4) node[pmos] (Mp1) {}
        (1,4) node[pmos] (Mp3) {}
        %WTA 2
        (6,-2) node[nmos] (Mb2) {}
        (4,0) node[nmos,xscale=-1] (M3) {}
        (M3) node[anchor=south east] {$M_3$}
        (6,0) node[anchor=south west] {$V_c$}
        (4,4) node[pmos] (Mp4) {}
        (6,2) node[nmos] (M4) {}
        (M4) node[anchor=west] {$M_4$}
        (6,4) node[pmos,xscale=-1] (Mp5) {}
        (8,4) node[pmos] (Mp6) {}
        %ground connectections
        (M1.source) node[ground] {}
        (Mb1.source) node[ground] {}
        (Mb2.source) node[ground] {}
        (M3.source) node[ground] {}
        %Vdd connections
        (Mp1.source) node[rrail] {}
        (Mp2.source) node[rrail] {}
        (Mp3.source) node[rrail] {}
        (Mp4.source) node[rrail] {}
        (Mp5.source) node[rrail] {}
        (Mp6.source) node[rrail] {}
        %inputs
        (Mb1.gate) node[anchor=south] {$V_b$}
        (Mb2.gate) node[anchor=south] {$V_b$}
        (Mp1.gate) node[anchor=south] {$V_{i1}$}
        (Mp4.gate) node[anchor=south] {$V_{i_2}$}
        %outputs
        (Mp3.drain) to[short,i=$I_{o1}$] ++(0,-0.1)
        (Mp6.drain) to[short,i=$I_{o2}$] ++(0,-0.1)
        %interconnect
        (Mp2.gate) to[short, *-] (Mp2.gate |- Mp2.drain) to[short,-*] (Mp2.drain)
        (Mp5.gate) to[short, *-] (Mp5.gate |- Mp5.drain) to[short,-*] (Mp5.drain)

        (Mp2.drain) to[short] (M2.drain)
        (Mp5.drain) to[short] (M4.drain)

        (Mp1.drain) to[short] (M1.drain)
        (Mp4.drain) to[short] (M3.drain)
        (M2.gate) to[short, -*] (M1.drain |- M2.gate)
        (M4.gate) to[short, -*] (M3.drain |- M4.gate)

        (M2.source) to[short] (Mb1.drain)
        (M4.source) to[short] (Mb2.drain)

        (M1.gate) to[short,-*] (-1,0) to[short,-*] (6,0)
        %intermediate signals
        (M2.gate) node[anchor=south] {$V_{o1}$}
        (M4.gate) node[anchor=south] {$V_{o2}$}
        (Mp1.drain) to[short,i=$I_{i1}$] ++(0,-0.1)
        (Mp4.drain) to[short,i=$I_{i2}$] ++(0,-0.1)
    ;\end{circuitikz}
    \caption{}
\end{figure}
When \(I_{i1}=I_{i2}\) it follows that \(I_{o1}=I_{o2}\) since for \(M_1\):
\begin{equation*}
    I_{i1} = I_0e^{\frac{\kappa V_c - 0}{U_T}}-I_0e^{\frac{\kappa V_c - V_{o1}}{U_T}}
\end{equation*}
And for \(M_2\)
\begin{equation*}
    I_{i2} = I_0e^{\frac{\kappa V_c - 0}{U_T}}-I_0e^{\frac{\kappa V_c - V_{o2}}{U_T}}
\end{equation*}
Where the second term in both equations is the reverse component and the first term is the forward component.
The only solution to the above equations are that \(V_{o1}=V_{o2}\) and therefore \(I_{o1}=I_{o2}=I_b\), 
meaning both \(M_2\) and \(M_4\) are conducting.

When \(I_{i1} \gg I_{i2}\), \(V_c\) must increase in order to cope with the increased current being forced through \(M_1\), causing
\(M_3\) to open further. However, the current through \(M_3\) is now very low compared to its gate voltage and therefore \(V_{o2}\) drops,
causing \(M_4\) to enter the deep triode region or even cutoff. Therefore 
\begin{align*}
    I_{o2} &\simeq 0 \\
    I_{o1} &\simeq 2I_b \\
    V_{o1} &\gg V_{o2} 
\end{align*}

This can be generalized to the \(n\)-input case. Here, for \(I_{i1}=I_{i2}=\ldots=I_{in}\) the output currents will be
\begin{equation*}
    I_{o1}=I_{o2}=\ldots=I_{on}=\frac{I_b}{n}
\end{equation*}
And for one input current \(I_{ix}\) much larger than any other input, then 
\begin{equation*}
    I_{ox} = nI_b
\end{equation*}

\subsection{Small Signal Model}
Applying a small differential input current, we have
\begin{equation*}
    I_i = I_u \pm \frac{1}{2}\Delta I_{i}
\end{equation*}
Because of the Early effect, the only way that transistors \(M_1\) and \(M_3\) can adapt to the changing current going through them
is to modify their drain voltages, assuming that \(V_c\) is constant (see appendix Fig. 1). The change in drain voltage is given by the drain conductance and
change in input current
\begin{equation*}
    \Delta V_o = \frac{\Delta I_i}{g_d}
\end{equation*}
Assuming positive \(\Delta I\) makes \(I_{i1}\) larger, this leads to
\begin{align*}
    I_{i1} = I_u+\frac{1}{2}\Delta I \quad & \quad I_{i2} = I_u-\frac{1}{2}\Delta I \\
                                           &\Downarrow \\
    V_{o1} = V_u+\frac{1}{2}\frac{\Delta I}{g_d} \quad & \quad V_{o2} = V_u-\frac{1}{2}\frac{\Delta I}{g_d} 
\end{align*}
As we are in subthreshold saturation, 
\begin{equation*}
    g_d = \frac{I_{sat}}{V_E}
\end{equation*}
And when using the bias current \(I_u\)
\begin{align*}
    I_{sat} \simeq \frac{I_u}{1+\frac{V_{ds}}{V_E}}
\end{align*}
However, since \(V_E\) is very large, this can be approximated as
\begin{equation*}
    I_{sat} \simeq I_u
\end{equation*}
And therefore
\begin{equation*}
    g_d = \frac{I_u}{V_E}
\end{equation*}

The change in output current is given by the change in output voltage and the gate transconductance of \(M_2\) and \(M_4\)
\begin{equation*}
    \Delta I_o = g_m\Delta V_o
\end{equation*}
Substituting \(\Delta V_o = \frac{\Delta I_i}{g_d}\) this becomes the input-output current relation is
\begin{equation*}
    \frac{\Delta I_{o}}{\Delta I_i} = \frac{g_m}{g_d}
\end{equation*}
Since each output current is biased at \(I_u=I_b\) the gate transconductance is
\begin{equation*}
    g_m = \frac{\kappa I_b}{U_T}
\end{equation*}
The input-output relation is then
\begin{equation*}
    \frac{\Delta I_o}{\Delta I_i} = \frac{\kappa I_b V_E}{I_u U_T}
\end{equation*}
And when normalized by \(I_b\) and \(I_i\)
\begin{equation*}
    \frac{\Delta I_o I_u}{\Delta I_i I_b} = \frac{\kappa V_E}{U_T}
\end{equation*}
Normal values of \(V_E\) are 150-750 V, \(\kappa\) is between 0.6-1 and \(U_T=26\) mV at room temperature. Using
\(V_E=450\) V and \(\kappa=0.8\) the normalized gain is
\begin{equation*}
    A_I = \frac{\Delta I_o I_u}{\Delta I_i I_b} = \frac{0.8\cdot450 V}{26\cdot 10^{-3} V} \simeq 14\cdot10^{3}
\end{equation*}

The assumption that \(V_c\) does not change holds for small differential currents because if \(V_c\) went up or down it would
try to force both the currents through \(M_1\) and \(M_4\) up or down with it according to the transconductance of the transistors.
This is impossible since we are applying a differential input current which forces one current up and the other down in relation 
to the common bias.

When operating above threshold, \(g_m\) for transistors \(M_2\) and \(M_4\) changes to 
\begin{equation*}
    g_m = \beta\left(\kappa\left(V_o-V_T\right)-V_c\right)
\end{equation*}
While the drain conductance stays the same.

\subsection{Draw the Setups}
See appendix.
\end{document}
