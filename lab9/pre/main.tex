\input{pre}
\tikzset{rrail/.style={rground,yscale=-1}}
\begin{document}
\input{frontpage}
\newpage
\section{Adaptive Elements}
See appended figures.

Fig.~1 Shows the circuit diagram and CMOS realisation of the tobi-element. When \(V_g > V_w\) current flows through B1 to ground
and through the diode formed by B2 to the well side. When \(V_w > V_g\) no current flows through any of the BJTs, but instead
through the diode-connected mosfet. 

Fig.~2 Shows the I-V curev of the tobi-element. When \(V_g>V_w\) the incoming current gets lost through the BJT going to ground
on the gate-coupled side of the mosfet. The current increases exponentially when the BJTs are operating as diodes, but saturates
when they hit their saturation limit. In the reverse case, the difference between the incoming and outgoing current is only due
to loss through the BJT couplings even though no current should ideally flow through them. The majority of the current goes through
the diode-connected PMOS transistor. Therefore the absolute current is lower.

\section{Photoreceptor Circuits}
Fig.~3 shows the output voltage of the SF receptor as a function of the logarithmic of the photocurrent which is directly proportional
to the irradiance. 

For very low intensities, the dark current dominates and the output is fixed. For higher intensities the output drops off linearly
with the log of the light intensity. For very high intensities, the bias transistor goes into the above-threshold regime and the
current becomes a quadratic function of the output voltage so it falls off more steeply.

\section{Receptor Gain}
\subsection{Small Signal Gain of SF and Adaptive Receptor}
In the last pre-lab we computed the gain of the SF receptor to be
\begin{equation*}
    A_{sf} = \frac{v_{out}}{i} = \frac{-U_T}{I_{ph}}
\end{equation*}
The normalised gain \(\hat{A}\) is then
\begin{equation*}
    \hat{A}_{sf} = -1
\end{equation*}
since \(I_{bg} \simeq I_{ph}\) for small signals.

To compute the gain of the adaptive photoreceptor we have to consider that it is a feedback system. 
Fig.~4 shows the general form of the feedback circuit that makes up the adaptive photoreceptor circuit as a block diagram.
It has the general solutions
\begin{align*}
    & y = \frac{acx}{ab+1} \\
    & e = \frac{-cx}{ab+1} 
\end{align*}
For the adaptive circuit 
\begin{align*}
    &y = v_{out} \\
    &x = i \\
    &c = \frac{U_T}{I} \\
    &a = A_{amp} \\
    b_{DC} = &\kappa  \; , \; b_{AC} = \kappa A_C
\end{align*}
The output \(v_{out}\) DC gain of a small signal \(i\) is then given by 
\begin{align*}
    &v_{out} = -A_{amp}\left(-\frac{U_T}{I}i - A_{amp}\kappa \right) \\
    & \Updownarrow \\
    &v_{out} = \frac{U_T}{I}\frac{A_{amp}}{\kappa A_{amp} + 1} \\
    & \Updownarrow \\
    & A_{aDC} = \frac{v_{out}}{i} =\frac{U_T}{I} \frac{A_{amp}}{\kappa A_{amp}+1} \\
    & \Downarrow \\
    & \hat{A}_{aDC} = \frac{A_{amp}}{\kappa A_{amp}+1}  \simeq \frac{1}{\kappa}
\end{align*}
Where the final step is justified because \(A_{amp}\gg 1\)

When considering the transient gain from \(v_{out}\) to \(v_{fb}\), the AC
gain of the closed loop is
\begin{align*}
    &v_{out} = -A_{amp}\left(-\frac{U_T}{I}i - A_{amp}\kappa A_C \right) \\
    & \Updownarrow \\
    &v_{out} = \frac{U_T}{I}\frac{A_{amp}}{\kappa A_C A_{amp} + 1} \\
    & \Updownarrow \\
    & A_a = \frac{v_{out}}{i} =\frac{U_T}{I} \frac{A_{amp}}{\kappa A_C A_{amp}+1} \\
    & \Downarrow \\
    & \hat{A}_a = \frac{A_{amp}}{\kappa A_C A_{amp}+1}  \simeq \frac{1}{\kappa A_C}
\end{align*}
Where \(A_C\) is given by 
\begin{equation*}
    A_C = \frac{C_2}{C_1+C_2}
\end{equation*}
Meaning that for \(C_1 = 10C_2\)
\begin{equation*}
    \hat{A}_a = \frac{11}{\kappa}
\end{equation*}


\subsection{Logarithmic Properties in Small-Signal Gain}
If 
\begin{equation*}
    y = \ln(x)
\end{equation*}
then
\begin{align*}
    &\frac{dy}{dx} = \frac{1}{x} \\
    &\Updownarrow \\
    &dy = \frac{dx}{x}
\end{align*}
Which in the case of circuits can be interpreted as a change around an operating point, which is
relative to the operating point itself. This shows up in the gains derived above, but disappears when normalising
by the background current.

\subsection{The Meaning of Small-Signal}
In the case of the photoreceptor circuit, the small changes in signal strength around a bias/operating point are a 
result of local changes in constrast around a common background irradiance.

\subsection{Forward Amplifier Gain}
The gain of the forward amplifier is the output resistance multiplied by the gate transconductance, because the input
appears on the gate. We have
\begin{equation*}
    A_{amp} = g_m R_{out}
\end{equation*}
The output resistance is the parallel combination of the N and P transistors drain resistance
\begin{equation*}
    g_{nd} || g_{pd} = g_{nd}+g_{np} = I\left(\frac{V_{EP}+V_{EN}}{V_{EP}V_{EN}}\right)
\end{equation*}
Which leads to the forward amplifier gain
\begin{equation*}
    A_{amp} = \frac{\kappa I}{U_T} \frac{V_{EN}V_{EP}}{I(V_{EN} + V_{EP})} = \frac{\kappa}{U_T} \frac{V_{EN}V_{EP}}{V_{EN} + V_{EP}}
\end{equation*}
For Early voltages of 20 V and a \(\kappa\) of 0.7 the gain is
\begin{equation*}
    A_{amp} = \frac{0.7}{26 \mathrm{mV}} \frac{20^2\mathrm{V}^2}{40 \mathrm{V}} = 269.23
\end{equation*}

\subsection{Input Node Response to Small-Signal Changes}
Again considering the closed loop configurations with and without the capacitive divider gain, the DC change in \(v_{in}\) is
\begin{equation*}
    \frac{v_{in}}{i} = -\frac{U_T}{I}\frac{1}{\kappa A +1}
\end{equation*}
And the AC gain
\begin{equation*}
    \frac{v_{in}}{i} = -\frac{U_T}{I}\frac{1}{\kappa A_C A +1}
\end{equation*}

\section{Time Response of the Source-Follower Receptor Circuit}
\subsection{Input Capacitance Time Constant}
The SF receptor circuit with an input capacitance can be transformed to an equivalent circuit where the diode is replaced by 
an equivalent resistance and the bias transistor by a current source. In that case, the s-domain equation is
\begin{equation*}
    V_s = \frac{I_dU_T}{I_p} \frac{1}{s\tau +1}
\end{equation*}
Where \(I_p\) is the bias transistor current and \(I_d\) is the diode current. Here,
\begin{equation*}
    \tau = \frac{CU_T}{I}
\end{equation*}

\subsection{Doubling the Capacitor Area}
Because
\begin{equation*}
    C = A\frac{\varepsilon}{d}
\end{equation*}
A doubling in area will double the time constant and thus slow down the circuit by a factor of two.

\section{Time Response of Adaptive Receptor}
\subsection{Total Loop Gain}
The total loop gain without an input capacitance was computed earlier to be
\begin{equation*}
    A_{loop} = \frac{v_{out}}{i} =\frac{U_T}{I} \frac{A_{amp}}{\kappa A_C A_{amp}+1} \\
\end{equation*}

\subsection{Total Loop Gain with Input Capacitance}
Adding a first-order lowpass filter to the forward path in the block diagram in Fig.~4 gives the following loop gain
\begin{equation*}
    A_{loop} = \frac{v_{out}}{i} =\frac{U_T}{I} \frac{A_{amp}}{\kappa A_C A_{amp}+\tau s + 1} \\
\end{equation*}
Where \(\tau\) is the time-constant computed in the previous section.
The denominator can be brought the the canonical low-pass filter form \(\tau s + 1\) as follows
\begin{align*}
    \kappa A_C A_{amp} + \tau s + 1 = 1 + \frac{\tau}{\kappa A_C A_{amp}}s + \frac{1}{\kappa A_C A_{amp}} \simeq 1 + \frac{\tau}{\kappa A_C A_{amp}}s
\end{align*}
So the new time constant is
\begin{equation*}
    \tau_{new} = \frac{\tau_{old}}{\kappa A_C A_{amp}}
\end{equation*}

\subsection{Adaptive Receptor Speedup}
The speedup of the adaptive receptor comes from the reduction in the time constant governing the input voltage. The speedup is
\begin{equation*}
    \mathrm{speedup} = \frac{\tau_{old}}{\tau_{new}} = \kappa A_C A_{amp}
\end{equation*}
Easily a factor of 1000 or more.


\end{document}
