\input{pre}

\tikzset{rrail/.style={rground,yscale=-1}}
\newcommand{\reffig}[1]{Fig.~\ref{#1}}

\begin{document}
\input{frontpage}
\newpage
\section{Post-Lab}
\subsection{BDJ Cross Section}
See appended figure.

\subsection{Electron-Hole Pair Generation in the BDJ}
Electron-hole pairs generated in the substrate will either recombine,
or the electron will diffuse into the n-well and contribute to the bottom current.

Pairs generated in the n-well can also recombine or diffuse into a depletion region.
Depending on which depletion region catches the hole, the pair generation will contribute to
either the bottom or top current.

The situation for the top p-region is the same as for the substrate, except any current will 
contribute to the top current.

\subsection{Spectral Response of the Top and Bottom Junction}
Intrinsic silicon has a peak quantum efficiency at about 800nm. However, these near infrared waves can
penetrate deep into the material. Therefore one would expect that the bottom junction of the BDJ
responds very well to long wavelengths, while the top junction absorbs the higher energy content of
the incoming light.

The response is sketched in the appended figures.

\subsection{Uses for Manufacturers and Drawbacks}
Since the response of the BDJ is widened, due to the increased absorbtion of higher wavelength light,
the BDJ can be uses as a wide-range image sensor in the visual spectrum.

Drawbacks include increased complexity of manufacturing, due to the extra well, and an increase in the dark
current due to having two junctions in a single diode.
\end{document}
