\input{pre}
\tikzset{rrail/.style={rground,yscale=-1}}
\begin{document}
\input{frontpage}
\newpage
\section{Log-Domain Pulse Integrator}
\subsection{Dynamical Equations}
Assuming that all transistors are operating in sub-threshold saturation, the following currents are flowing
\begin{align*}
    &I_\tau = I_0e^{\frac{\kappa(V_{dd}-V_\tau)}{U_T}} \\
    &I_w = I_0e^{\frac{\kappa(V_s-V_w)}{U_T}} = I_\tau + I_c \\
    &I_c = -C\frac{dV_s}{dt} \\
    &I_{syn} = I_0e^{\frac{\kappa(V_{dd}-V_s)}{U_T}}
\end{align*}
By replacing
\begin{equation*}
    I_w = I_0e^{\frac{\kappa(V_{dd}-V_w)-\kappa(V_{dd}-V_s)}{U_T}}  = \frac{I_{w0}}{I_{syn}}
\end{equation*}
We have
\begin{align*}
    &\frac{dV_s}{dt} = \frac{I_\tau - \frac{I_0I_{w0}}{I_{syn}}}{C} \\
    &\Downarrow \\
    &\frac{dI_s}{dt} = \frac{-\kappa\left(I_\tau - \frac{I_0I_{w0}}{I_{syn}}\right)}{U_TC}I_{syn} \\
    &\Updownarrow \\
    &\frac{CU_T}{\kappa I_\tau}\frac{dI_{syn}}{dt} + I_{syn} = \frac{I_0I_{w0}}{I_\tau}
\end{align*}

\subsection{Rising and Falling Step Response}
Transforming the above result to the Laplace domain reveals
\begin{equation*}
    I_{syn} = \frac{I_0I_{w0}}{I_\tau}\frac{1}{\frac{CU_T}{\kappa I_\tau}s+1}
\end{equation*}
The second fraction in this equation is equivalent to a canonical low-pass filter with 
\begin{equation*}
    \tau = \frac{CU_T}{\kappa I_\tau}
\end{equation*}
The solutions to up and down-going steps starting from minimum and maximum charge on the capacitor are
then
\begin{equation*}
    I_{syn} = \frac{I_0I_{w0}}{I_\tau}\left(1-e^{\frac{-t}{\tau}}\right)
\end{equation*}
For the rising edge case, and
\begin{equation*}
    I_{syn} = \frac{I_0I_{w0}}{I_\tau}e^{\frac{-t}{\tau}}
\end{equation*}
For the falling edge case, assuming that both edges occur at t=0.

\subsection{\(V_{syn}\) During a Spike Train}
When the input is a spike train with frequency \(f\), where each spike has width \(w\), the steady state depends
on both the frequency and width of each spike. 

When a spike arrives, \(V_{syn}\) will drop because the capacitor holding the voltage is being discharged through the
weight and input transistors. When the spike stops, the capacitor will start charging again and thus bring \(V_{syn}\) back up.

In the steady state, \(V_{syn}\) will be oscillation around some DC value determined by the input frequency.

\subsection{Relation of Spike Response to Step Response}
For the step response, \(I_w\) determines the current carried by the synapse and \(I_\tau\) determines the time constant of the 
synapse. Modifying these values will show up as a difference in the DC level of the output during an incoming spike train, as well
as a change in the maximum allowable frequency that does not result in saturation.
    
\end{document}
