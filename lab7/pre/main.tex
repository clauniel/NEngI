\input{pre}
\tikzset{rrail/.style={rground,yscale=-1}}
\begin{document}
\input{frontpage}
\newpage
\section{Pre-Lab}
\subsection{CMOS Capacitors}
In CMOS technology, capacitors can be constructed by depositing an \(\mathrm{SiO_2}\) layer on top of a wire and
adding a wire to the top of the layer. This effective creates a capacitor with the same thickness as the oxide layer.
Another method is to use a normal well MOS transistor without source and drain. The bulk and gate terminals then
act as the terminals of a capacitor. The problem with this method is that the capacitance of this type of capacitor 
will change with the applied voltage because a depletion layer will be formed in the bulk substrate.

The relative permittivity for \(\mathrm{SiO_2}\) is 3.9 (Wilk, Wallace, Anthony 2001). A 10 nm thick capacitor with this dieletric
material will therefore have a capacitance per area of 
\begin{equation*}
    \frac{C}{A} = \frac{3.9\varepsilon_0}{10nm} = 3.45 \frac{\mathrm{fF}}{\mu\mathrm{m}}
\end{equation*}

For the lipid bilayer, the permittivity varies with thicking. For a thickness of 5 nm the relative permittivity is between 1.74 and 1.51 (Ohki, Shinpei 1967),
and the layer will therefore have the capacitance between
\begin{equation*}
    \frac{C}{A} = \frac{1.74\varepsilon_0}{5\mathrm{nm}} = 3.08 \frac{\mathrm{fF}}{\mu\mathrm{m}}
\end{equation*}
and
\begin{equation*}
    \frac{C}{A} = \frac{1.51\varepsilon_0}{5\mathrm{nm}} = 2.67 \frac{\mathrm{fF}}{\mu\mathrm{m}}
\end{equation*}

\subsection{Follower-Integrator Transfer Function}
The output current of the transconductance amplifier is
\begin{equation*}
    I_{out} = I_b\tanh\left(\frac{\kappa \left( V_+ - V_- \right)}{2U_T}\right)
\end{equation*}
Which in the case of the follower-integrator circuit is equivalent to 
\begin{equation*}
    I_{out} = I_b\tanh\left(\frac{\kappa \left( V_{in} - V_{out} \right)}{2U_T}\right)
\end{equation*}
We assume that we are operating in the linear range where \(\tanh(x) \simeq x\) so that
\begin{equation*}
    I_{out} = \frac{\kappa I_b}{2U_T}\left(V_{in} - V_{out}\right) = G\left(V_{in} - V_{out}\right)
\end{equation*}
Where \(G\) is the transconductance of the amplifier.

Since no current flows into the input terminals of the transconductance amplifier, all of the
output current flows through the capacitor, therefore
\begin{equation*}
    I_{out} = C\frac{dV_{out}}{dt}
\end{equation*}
Equation the above expressions
\begin{align*}
    &C\frac{dV_{out}}{dt} = G\left(V_{in} - V_{out}\right) \\
    &                     \Updownarrow \\
    &\frac{C}{G}\frac{dV_{out}}{dt} + V_{out} =  V_{in} \\
\end{align*}
Replacing \(\tau = \frac{C}{G}\) and transforming to the Laplace domain this becomes
\begin{align*}
    &V_{out}\left( s\tau + 1 \right) = V_{in} \\
    &\Downarrow  \\
    &H(s) = \frac{V_{out}}{V_{in}} = \frac{1}{1+s\tau}
\end{align*}

\subsection{Follower-Differentiator Transfer Function}
For the follower-differentiator, the time-domain equations are
\begin{align*}
    &C\frac{d}{dt}\left(V_{in}-V_{out}\right) = GV_{out} \\
    &\Updownarrow \\
    &\tau \frac{dV_{out}}{dt} + V_{out} = \tau \frac{dV_{in}}{dt} 
\end{align*}
And by transformation to the Laplace-domain
\begin{align*}
    &V_{out}\left(s\tau + 1\right) = s\tau V_{in} \\
    &\Downarrow \\
    &H(s) = \frac{V_{out}}{V_{in}} = \frac{s\tau}{1+s\tau}
\end{align*}

\subsection{Half-Power Frequency}
Since the ideal capacitor is a component having only imaginary resistance, reactance, the \(\sigma\) in \(s=\sigma+j\omega\) is \(0\) and so
\begin{equation*}
    H(j\omega) = \frac{1}{1+j\tau \omega}
\end{equation*}
For the follower-integrator.

The half-power frequency is when \(|H(jw)|=\frac{1}{\sqrt{2}}\) because power scales quadratically with voltage, solving for \(\omega\) we have
\begin{align*}
    |H(j\omega)| = \frac{1}{|1+j\tau\omega|} &= \frac{1}{\sqrt{1^2+\tau^2\omega^2}} = \frac{1}{\sqrt{2}} \\
    &\Downarrow \\
    &\omega = \frac{1}{\tau}
\end{align*}
As \(\omega = 2\pi f\) the half-power frequency is
\begin{equation*}
    f = \frac{1}{2\pi\tau}
\end{equation*}

\subsection{Comparing RC Integrators}
The simple RC integrator consists of ideal components whose behaviour is well defined over all input ranges. When deriving the transfer function
for the transconductance amplifier follower-integrator we assumed that we were operating in the linear range which demands that the input voltages
are not too dissimilar. If the differential input were to become too large, the output current would become constant with respect to the differential
input voltage. This is not consistent with the derived transfer function and essentially "breaks" the model. 

When the differential input becomes too large for the circuit to handle, it is said to be slew-rate limited.

\subsection{Small-Signal Model}
The meaning of "small-signal" depends on a model of circuit operation. Typically a small-signal model describes the way a circuit behaves for 
small-signals around an operating point, where the meaning of "small" typically depends on a linearisation of a more complex I-V relationship for
a given component.

For the transconductance amplifier, the magnitude of the differential input is the main restriction. Since we assumed that \(I_{out}=I_b\tanh\left(\frac{\kappa \Delta V}{2U_T}\right)\)
in the derivation of the transfer function, which is only approximately true for \(2U_T<\Delta V < 2U_T\).
The linearity criterion for the follower-integrator is
\begin{equation*}
    \frac{dV_{in}}{dt} < \frac{4U_T}{\tau}
\end{equation*}

The absolute magnitude of the input also matters since the current mirror transistors must be in saturation and the input transistors in sub-threshold.
\end{document}
