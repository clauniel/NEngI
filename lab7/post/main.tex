\documentclass[a4paper]{article}
\usepackage[utf8]{inputenc} % Skal passe til editorens indstillinger
\usepackage[english]{babel} % danske overskrifter


\newcommand{\name}{Carsten Nielsen}
%\newcommand{\stnumber}{s123369, s123161, s123821}
\newcommand{\course}{INI 404 Neuromorphic Engineering~I}
\newcommand{\university}{University of Zürich}
\newcommand{\studyline}{Institute of Neuroinformatics}
\newcommand{\assignment}{Lab 8 Post-Lab}
\renewcommand{\date}{\today} %If another date, than that of today is desiered


% Palatino for rm and math | Helvetica for ss | Courier for tt
\usepackage{mathpazo} % math & rm
\linespread{1.05}        % Palatino needs more leading (space between lines)
\usepackage{palatino} % tt
\normalfont
\usepackage[T1]{fontenc}
\usepackage[english]{babel}

\usepackage{graphicx}%allerese hentet % indsættelse af billeder
\usepackage{epstopdf} %Tilfj "--enable-write18" i argumentet for LaTex build. Dette vil konvertere .eps figurer til pdf-format
\graphicspath{{./picture/}} % stivej til bibliotek med figurer
\usepackage{subcaption} %Til gruppering af figurer
\usepackage{amsmath} %matpakke
\usepackage{amsfonts} %
\usepackage{amssymb} %
\usepackage{steinmetz} % flere matematik symboler
\usepackage{polynom} %for displaying polynom division
\usepackage{mathtools} % matematik - understøtter muligheden for at bruge \eqref{}
\usepackage{float}
\usepackage{placeins}
\usepackage{hhline}

%
\usepackage[usenames,dvipsnames]{xcolor}
\usepackage[compact,explicit]{titlesec}% http://ctan.org/pkg/titlesec
%
\usepackage[europeanresistors]{circuitikz}
\usepackage{pgfplots}
\usepgfplotslibrary{patchplots}
\pgfplotsset{compat=1.11}

%---------%
%Easy edit%
%---------%

%Section formating. arg1 is supplied when making section
\newcommand\presectionnumber[1]{~~}
\newcommand\postsectionnumber[1]{}
\newcommand\midlesection[1]{#1}
\newcommand\sectionnum[1]{\arabic{#1}}
\newcommand\subsectionnum[1]{\arabic{#1}}
\newcommand\subsubsectionnum[1]{\alph{#1}}



%------------%
%setion setup%
%------------%
\renewcommand\thesection{Opgave~\sectionnum{section}} %pas p�, kun i matematik
\renewcommand\thesubsection{\thesection,~\subsectionnum{subsection}}
\definecolor{MagRed}{RGB}{190,40,15}
\definecolor{MathGreen}{RGB}{82,164,0}

\titleformat{\section}{\normalfont\sffamily\large\bfseries\color{MathGreen}}{}{0pt}{|\kern-0.15ex|\kern-0.15ex|\kern-0.15ex|\presectionnumber{#1}\sectionnum{section}\postsectionnumber{#1}\qquad\quad\midlesection{#1}\label{sec:\sectionnum{section}}}
\titleformat{\subsection}[runin]{\large\bfseries}{}{10pt}{\sectionnum{section}.\subsectionnum{subsection})~#1\label{sec:\sectionnum{section}.\subsectionnum{subsection}}}
\titleformat{\subsubsection}[runin]{\itshape}{}{0pt}{\subsectionnum{subsection},\subsubsectionnum{subsection}~#1\label{sec:\sectionnum{section}.\subsectionnum{subsection}.\subsubsectionnum{subsubsection}}}
%\titleformat{\subsubsection}{\bfseries}{}{0pt}{\alph{subsection}.\arabic{subsubsection})\qquad\quad#1\label{\arabic{section}\alph{subsection}\arabic{subsubsection}}}

%----------%
%page setup%
%----------%
\textwidth = 400pt
\marginparwidth = 86pt
\hoffset = -25pt
\voffset= -30pt
\textheight = 670pt

%--------%
%hyperref%
%--------%
\newcommand{\HRule}{\rule{\linewidth}{0.5mm}}
\usepackage{fancyhdr}
\usepackage[plainpages=false,pdfpagelabels,pageanchor=false]{hyperref} % aktive links
\hypersetup{%
  pdfauthor={\name},
  pdftitle={\assignment},
  pdfsubject={\course} }
%\usepackage{memhfixc}% rettelser til hyperref

%-------------%
%Headder setup%
%-------------%
\fancyhf{} % tom header/footer
\fancyhfoffset{20pt}
\fancyhfoffset{20pt}
\fancyhead[OL]{\name \\ INI 404}
\fancyhead[OC]{Date \\ \date}
\fancyhead[OR]{\university\\ \studyline}
\fancyfoot[FL]{}
\fancyfoot[FC]{\thepage}
\fancyfoot[FR]{}
\renewcommand{\headrulewidth}{0.4pt}
\renewcommand{\footrulewidth}{0.4pt}
\headsep = 35pt
\pagestyle{fancy}
 % style setup

%Listings%
\usepackage{listingsutf8}
\usepackage[framed,numbered]{matlab-prettifier}


%setup listings
\lstset{language=Matlab,
  extendedchars=true,
  language=Octave,                % the language of the code
  basicstyle=\ttfamily\footnotesize,           % the size of the fonts that are
  % used for the code
  numbers=left,                   % where to put the line-numbers
  numberstyle=\tiny\color{gray},  % the style that is used for the line-numbers
  stepnumber=2,                   % the step between two line-numbers. If it's 1, each line 
                                  % will be numbered
  numbersep=5pt,                  % how far the line-numbers are from the code
  backgroundcolor=\color{white},      % choose the background color. You must add \usepackage{color}
  showspaces=false,               % show spaces adding particular underscores
  showstringspaces=false,         % underline spaces within strings
  showtabs=false,                 % show tabs within strings adding particular underscores
  frame=single,                   % adds a frame around the code
  rulecolor=\color{black},        % if not set, the frame-color may be changed on line-breaks within not-black text (e.g. comments (green here))
  tabsize=4,                      % sets default tabsize to 2 spaces
  captionpos=b,                   % sets the caption-position to bottom
  breaklines=true,                % sets automatic line breaking
  breakatwhitespace=false,        % sets if automatic breaks should only happen at whitespace
  title=\lstname,                   % show the filename of files included with \lstinputlisting;
                                  % also try caption instead of title
  %keywordstyle=\color{blue},          % keyword style
  %commentstyle=\color{dkgreen},       % comment style
  %stringstyle=\color{mauve},         % string literal style
  escapeinside={\%*}{*)},            % if you want to add LaTeX within your code
  morekeywords={*,...},              % if you want to add more keywords to the set
  deletekeywords={...}              % if you want to delete keywords from the given language
}
\lstset{literate=
  {á}{{\'a}}1 {é}{{\'e}}1 {í}{{\'i}}1 {ó}{{\'o}}1 {ú}{{\'u}}1
  {Á}{{\'A}}1 {É}{{\'E}}1 {Í}{{\'I}}1 {Ó}{{\'O}}1 {Ú}{{\'U}}1
  {à}{{\`a}}1 {è}{{\`e}}1 {ì}{{\`i}}1 {ò}{{\`o}}1 {ù}{{\`u}}1
  {À}{{\`A}}1 {È}{{\'E}}1 {Ì}{{\`I}}1 {Ò}{{\`O}}1 {Ù}{{\`U}}1
  {ä}{{\"a}}1 {ë}{{\"e}}1 {ï}{{\"i}}1 {ö}{{\"o}}1 {ü}{{\"u}}1
  {Ä}{{\"A}}1 {Ë}{{\"E}}1 {Ï}{{\"I}}1 {Ö}{{\"O}}1 {Ü}{{\"U}}1
  {â}{{\^a}}1 {ê}{{\^e}}1 {î}{{\^i}}1 {ô}{{\^o}}1 {û}{{\^u}}1
  {Â}{{\^A}}1 {Ê}{{\^E}}1 {Î}{{\^I}}1 {Ô}{{\^O}}1 {Û}{{\^U}}1
  {œ}{{\oe}}1 {Œ}{{\OE}}1 {æ}{{\ae}}1 {Æ}{{\AE}}1 {ß}{{\ss}}1
  {ç}{{\c c}}1 {Ç}{{\c C}}1 {ø}{{\o}}1 {å}{{\r a}}1 {Å}{{\r A}}1
  {€}{{\EUR}}1 {£}{{\pounds}}1
}

 \lstloadlanguages{% Check Dokumentation for further languages ...
         %[Visual]Basic
         %Pascal
         %C
         %C++
         %XML
         %HTML
         %Java
         %VHDL
         Matlab
 }
 %Listings slut%









%Matematik hurtige ting
%fed
\renewcommand\vec[1]{\mathbf{#1}}
\newcommand\matr[3]{{}_{#2}\mathbf{#1}{}_{#3}}
\newcommand\facit[1]{\underline{\underline{#1}}}
%\renewcommand\d[3]{\frac{\mbox{d}^{#3}#1(#2)}{\mbox{d}#2^{#3}}}
%underline
%\renewcommand\vec[1]{\underline{#1}}
%\newcommand\matr[3]{{}_{#2}\underline{\underline{#1}}{}_{#3}}

\renewcommand\matrix[4]{ %{alignment}{to space}{from space}{matrix}
{\vphantom{\left[\begin{array}{#1}#4\end{array}\right]}}_{#2}\kern-0.5ex
\left[\begin{array}{#1}
#4
\end{array}\right]_{#3}
}
\newcommand\e[0]{\mbox{e}}
\newcommand\E[1]{\cdot 10^{#1}}
\newcommand\im[0]{i}

\newcommand\Jaco{\mbox{Jacobi}}
\newcommand\del[2]{\frac{\partial {#1}}{\partial {#2}}}
\newcommand\abs[1]{\left| {#1} \right|}
\newcommand\stdfig[4]{ %width,img,cap,lab
\begin{figure}[H]
\centering
\includegraphics[width={#1}\textwidth]{#2}
\caption{#3}
\label{#4}
\end{figure}
}
\newcommand\stdfignoscale[3]{ %img,cap,lab
\begin{figure}[H]
\centering
\includegraphics{#1}
\caption{#2}
\label{#3}
\end{figure}
}
\newcommand\diff{\dot}
\newcommand\ddiff{\ddot}
\newcommand\dddiff{\dddot}
\newcommand\ddddiff{\ddddot}






% How to make ref to books or urls in bib
%\citetitle[fx: page 1]{name of ref in bib}


\tikzset{rrail/.style={rground,yscale=-1}}
\newcommand{\reffig}[1]{Fig.~\ref{#1}}

\begin{document}
\section{Post-Lab}
\subsection{Continuous-time RC Integrator Frequency Range}
\begin{figure}
    \begin{subfigure}{0.5\textwidth}
        \center
        \begin{circuitikz}[american resistors, scale=1.4]\draw
            (0,0) to[R=$R$,o-] (1,0) to[C=$C$] (1,-1) to[short,-o] (0,-1) to[open, v^>=$V_{in}$] (0,0)
            (1,0) to[short,-o] (2,0) to[open, v^=$V_{out}$] (2,-1) to[short, o-] (1,-1)
        ;\end{circuitikz}
        \caption{RC low-pass filter (integrator) circuit.}
    \end{subfigure}
    \begin{subfigure}{0.5\textwidth}
        \center
        \begin{circuitikz}[american resistors, scale=1.4]\draw
            (0,0) to[C=$C$,o-] (1,0) to[R=$R$] (1,-1) to[short,-o] (0,-1) to[open, v^>=$V_{in}$] (0,0)
            (1,0) to[short,-o] (2,0) to[open, v^=$V_{out}$] (2,-1) to[short, o-] (1,-1)
        ;\end{circuitikz}
        \caption{RC high-pass filter (differentiator) circuit.}
    \end{subfigure}

    \begin{subfigure}{0.5\textwidth}
        \center
        \begin{circuitikz}[american resistors, scale=1.4]\draw
            (0,0) node[op amp] (transamp) {}
            (transamp.down) to[short,-o] ++(0,-0.5) node[anchor=north] {$V_b$}
            (transamp.-) to[short] ++(0,0.5) node (fback) {} to[short] (fback -| transamp.out) to[short] (transamp.out)
            (transamp.-) to[short, *-o] ++(-1,0) node[anchor=east] {$V_{bDC}$}
            (transamp.+) to[short, -o] ++(-1,0) node[anchor=east] {$V_{in}$}
            (transamp.out) to[C=$C$,i=$I_{out}$] ++(0,-1) node[ground] {}
            (transamp.out) to[short, *-o] ++(0.5,0) node[anchor=south] {$V_{out}$}
        ;\end{circuitikz}
        \caption{Follower-integrator (low-pass filter).}
    \end{subfigure}
    \begin{subfigure}{0.5\textwidth}
        \center
        \begin{circuitikz}[american resistors, scale=1.4]\draw
            (0,0) node[op amp] (transamp) {}
            (transamp.down) to[short,-o] ++(0,-0.5) node[anchor=north] {$V_b$}
            (transamp.-) to[short] ++(0,0.5) node (fback) {} to[short, i=$I_{out}$] (fback -| transamp.out) to[short] (transamp.out)
            (transamp.-) to[C=$C$, *-o] ++(-1,0) node[anchor=east] {$V_{in}$}
            (transamp.+) to[short] ++(-1,0) node[ground] {}
            (transamp.out) to[short, *-o] ++(0.5,0) node[anchor=south] {$V_{out}$}
        ;\end{circuitikz}
        \caption{Follower-differentiator (high-pass filter).}
    \end{subfigure}

    \begin{subfigure}{\textwidth}
        \center
        \begin{circuitikz}[american resistors, scale=1.4]\draw
            %HP component
            (-4,0) node[op amp] (transamp) {}
            (transamp.down) to[short,-o] ++(0,-0.5) node[anchor=north] {$V_b$}
            (transamp.-) to[short] ++(0,0.5) node (fback) {} to[short, i=$I_{out}$] (fback -| transamp.out) to[short] (transamp.out)
            (transamp.-) to[C=$C1$, *-o] ++(-1,0) node[anchor=east] {$V_{in}$}
            (transamp.+) to[short] ++(-1,0) node[ground] {}
            (transamp.out) to[short, *-] ++(0.75,0) node (out1) {}
            % 
            %LP component
            (0,0) node[op amp] (transamp) {}
            (transamp.down) to[short,-o] ++(0,-0.5) node[anchor=north] {$V_b$}
            (transamp.-) to[short] ++(0,0.5) node (fback) {} to[short] (fback -| transamp.out) to[short] (transamp.out)
            (transamp.-) to[short, *-] ++(-0.5,0) node[ground] {}
            (transamp.+) to[short] (out1 |- transamp.+) node[anchor=east] {} to[short] (out1)
            (transamp.out) to[C=$C2$,i=$I_{out}$] ++(0,-1) node[ground] {}
            (transamp.out) to[short, *-o] ++(0.5,0) node[anchor=south] {$V_{out}$}
        ;\end{circuitikz}
        \caption{Bandpass filter using two transconductance amplifiers. Note that the ground symbol does not necessarily represent 0V since that
        would put the repsective input transistors into cutoff.}
    \end{subfigure}
    \caption{Circuits considered in this lab.}
    \label{fig:1}
\end{figure}
The low-pass filter circuit in Fig.~\ref{fig:1}(a) is not a true integrator. However, when the voltage of the capacitor is far from the input voltage,
the exponential part of the step response is approximately linear and the circuit therefore acts as an approximate integrator.
For this to always be true, the input voltage must switch so quickly that the capacitor cannot fully charge or discharge.
When a step input is applied at \(t=0\), the voltage on the capacitor is approximately linear with respect to time for 
\begin{equation*}
    0<t<\tau
\end{equation*}
Where \(\tau=RC\). Since
\begin{equation*}
    \tau = \frac{1}{2\pi f_0}
\end{equation*}
for an ideal first order low-pass filter, it follows that if
\begin{equation*}
    f > \frac{1}{2\pi\tau} = f_0
\end{equation*}
Then the output will always behave approximately linear as an integrator.

For frequencies outside this range, the voltage across the capacitor will stop increasing linearly and saturate at \(V_c = V_{in}\).

\subsection{Asymmetry of Slew-Rates}
The difference in the up and downgoing slew-rates is caused by transistor mismatch in the differential pair and current mirrors of the 
transconductance amplifier. The mismatch is always there and so one would also expect it to show up during linear operation.

The mismatch causes an offset in the 0 output current and also in the minimum and maximum values of output current. With no mismatch
the output current moves from \(-I_b\) to \(I_b\), but with mismatch slightly more current will flow for one sign of the differential 
input voltage. This is true both in the linear and saturation regions.

\subsection{Asymmetry of Slew-Rates Continued}
The DC offset of the output signal will be different from the DC offset of the input as noted, again because of the non-zero differential voltage
offset that leads to a zero output current. 

Because the rise and fall times are different, the output will spend more time at either the high or low extreme of its values, thereby driving
the average up or down relative to the input. If the rise time is smaller than the fall time, the circuit will spend more time at the high
output extreme, driving the output offset up. Conversely the output DC offset will be driven down when the fall time is smaller than the 
rise time.

\subsection{The Follower-Integrator in Above-Threshold Operation}
Our small signal model rests on the linearisation around \(\Delta V = 0\) of the output current
\begin{equation*}
    I_{out} = I_b\tanh\left(\frac{\kappa\Delta V}{2U_T}\right)
\end{equation*}
Yielding the conductance 
\begin{equation*}
    g_m = \frac{\kappa I_b}{2U_T}
\end{equation*}

When the circuit is operating above threshold, the output current is given by 
\begin{equation*}
    I_{out} = \frac{\beta}{2}\Delta V\sqrt{\frac{4I_b}{\beta}-\Delta V^2}
\end{equation*}
For \(\Delta V < V_b-V_{T0}\), which keeps the square root term positive. Beyond this the current saturates.

For small signals this can be linearised around \(\Delta V = 0\) to yield the conductance
\begin{equation*}
    g_m = \sqrt{I_b\beta}
\end{equation*}
The width of the linear region is now dependent on \(I_b\) and by extension \(V_b\). The slew-rate stays the same since the current
still saturates to \(\pm I_b\). Beyond this the model is the same except for the variable linear range and change in time constants.

\subsection{The Follower-Differentiator}
The follower-differentiator and the corresponding RC circuit are shown in Fig.~\ref{fig:1}(d) and (b) respectively.

For the differentiator to actually differentiate, requires that the speed with which the input changes is so low that the capacitor is
always approximately fully charged. This way the following holds
\begin{equation*}
    \frac{dV_{in}}{dt} = \frac{dV_c}{dt}
\end{equation*}
and the circuit functions as a true differentiator.

When a step input is applied to the follower-differentiator, the output voltage will approximately have reached its final value after \(4\tau\)
and so as long as the signal changes with a frequency which is so low that
\begin{equation*}
    f < \frac{1}{8\pi\tau}
\end{equation*}
the circuit will act as a differentiator.

The consequence of not obeying this requirement can be seen in the example of a step input being applied to the differentiator. Just after the step is
applied, the output will have the same value as before the step, which is naturally not the differentiated value of the step input.

\subsection{A Band-Pass Filter}
Fig.~\ref{fig:1}(e) shows a band-pass filter constructed using two transconductance amplifiers. The high-pass filter is placed first in the 
cascade to remove any unwanted DC offset other than the reference offset denoted by ground.

The band-pass filter is constructed using identical transconductance amplifiers, so the conductances for small signals are equal. The transfer
function of the whole filter is composed of the two individual transfer functions.
\begin{equation*}
    H(s) = H_{hp}(s)H_{lp}(s) = \frac{\tau_{hp}s}{1+\tau_{hp}s}\cdot\frac{1}{1+\tau_{lp}s}
\end{equation*}
Where
\begin{equation*}
    \tau_{hp} = \frac{C_1}{G} \;,\;\; \tau_{lp} = \frac{C_2}{G}
\end{equation*}
The cutoff frequencies will then be
\begin{equation*}
    f_{0l} = \frac{1}{2\pi\tau_{lp}} \;,\;\; f_{0h} = \frac{1}{2\pi\tau_{hp}}
\end{equation*}
\end{document}
