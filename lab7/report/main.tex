\input{pre}

\tikzset{rrail/.style={rground,yscale=-1}}
\newcommand{\reffig}[1]{Fig.~\ref{#1}}

\begin{document}
\input{frontpage}
\newpage
\begin{figure}[!htb]
    \center
    \includegraphics{ex1-step.eps}
    \caption{Square wave input to the RC circuit and the output signal. The time constant is shown for the falling edge.}
    \label{fig:ex1-1}
\end{figure}
\begin{figure}[!htb]
    \center
    \includegraphics{ex1-log.eps}
    \caption{The difference between the final and immediate values of the output voltage during a discharge with a line fit to find \(\tau\).}
    \label{fig:ex1-2}
\end{figure}
\section{The RC Integrator}
We built an RC circuit using a 22 kOhm resistor and 47 nF capacitor which theoretically yields a time constant of
\begin{equation*}
    \tau = RC = 0.1034 \mathrm{ms}
\end{equation*}
Fig.~\ref{fig:ex1-1} shows the input and outputs of the circuit when a 100 mV PP square wave with a frequency of 1 kHz is applied at the input around
a DC bias voltage of 2 V. We find the actual time constant of the circuit by fitting a line to the logarithm of the difference between the 
final output value and the immediate value as seen in Fig.~\ref{fig:ex1-2}. This is useful because 
\begin{equation*}
V(t)=V_\infty(1-e^{-t/\tau})
\end{equation*}
and so
\begin{equation*}
log(V(t)-V_\infty) = -t/\tau + log(V_\infty)
\end{equation*}

We find a time constant of 0.1003 ms which is well within the tolerance of the components used to construct the circuit.
\begin{figure}[!htb]
    \center
    \includegraphics{ex2-step.eps}
    \caption{Square wave input to the RC circuit and the output signal. The output reaches a higher voltage than the input square wave
    because of the non-zero offset voltage for 0 output current.}
    \label{fig:ex2-1}
\end{figure}
\begin{figure}[!htb]
    \center
    \includegraphics{ex2-log.eps}
    \caption{The difference between the final and immediate values of the output voltage during charge and discharge of the output capacitor. 
    Lines are fit to the rising and falling edges to find \(\tau_r\) and \(\tau_f\) respectively.}
    \label{fig:ex2-2}
\end{figure}
\section{Time-Domain Response of Follower-Integrator}
Fig.~\ref{fig:ex2-1} shows the output of the follow-integrator when the input is driven by a 50 mV PP square wave with a frequency of 5 kHz biased around
2 V DC. Fig.~\ref{fig:ex2-2} shows the log of the difference between the target voltage for the output and the immediate value.
The falling and rise edge time constants are found by fitting lines to these curves.

We find that the rising time constant is slightly smaller than the falling one
\begin{equation*}
    \tau_r = 6.96\mu\mathrm{s} \; \tau_f = 7.45\mu\mathrm{s}
\end{equation*}

This means that the current flows out of the amplifier and into the capacitor faster than it does in the opposite direction. A mismatch in the differential pair would only produce an offset between input and output, but the slope of the \(I_{out}\) curve around the operation point and therefore the input and output currents would be equal. 
Instead, the effect must be caused by a mismatch in the current mirror so that it makes a larger copy of the current. As we discussed in Lab 4, this makes the positive saturation value of \(I_{out}\) larger than the negative one, therefore making the slope of the \(I_{out}\) curve larger for a wider range in the positive part than in the negative one. This in turns explains why the current flows more quickly in the outward (positive) direction. 


\section{Frequency-Domain Response of the Follower-Integrator}
\begin{figure}
    \center
    \includegraphics{ex3-freqresp.eps}
    \caption{Frequency response of the follower integrator and a theoretical ideal first-order low-pass filter. The cutoff frequency is 23.5 kHz.}
    \label{fig:ex3-1}
\end{figure}
Fig.~\ref{fig:ex3-1} shows our measurements of the frequency response for the follower-integrator. We first honed in on the cutoff frequency manually by finding the frequency for which the output amplitude drops by \(1/\sqrt 2\). 
After this, we took logarithmically spaced measurements of the input-output magnitude. The line is the frequency response for a first order low-pass
filter with a cutoff frequency of 23.5 kHz. The measurements drop off more quickly, indicating some second order capacitive effects in the circuit,
stemming from parasitic capacitances and inductances in the breadbord and connecting wires.

A cutoff frequency of 23.5 kHz, implies that the time constant of the circuit is
\begin{equation*}
    \tau = \frac{1}{2\pi F_0} = 6.77 \mu\mathrm{s}
\end{equation*}
Which is slightly below the value obtained in experiment 2. However, in experiment 2, the frequency was 5 kHz, while the cutoff frequency here is 23.5 kHz which
may influence the rise and fall times for a non-ideal circuit.

\section{Large Signal Behaviour of the Follower-Integrator}
\subsection{Slew-Rate Limiting}
\begin{figure}
    \center
    \includegraphics{ex4-slew.eps}
    \caption{Large signal input-output relation in the time domain for the follower-integrator. The dotted lines represent the limits of up and downgoing slew-rate limited operation.}
    \label{fig:ex4-1}
\end{figure}
Fig.~\ref{fig:ex4-1} shows a large signal square wave applied to the follower integrator. With this input, the circuit is slew-rate limited
and the charging of the capacitor proceeds linearly until the differential voltage is a small signal again.

The slew-rate time constants for the upgoing and downgoing edges are
\begin{equation*}
    \tau_r = 10.22\frac{\mathrm{kV}}{\mathrm{s}} \; \tau_f = 12.04\frac{\mathrm{kV}}{\mathrm{s}}
\end{equation*}
And their ratio 
\begin{equation*}
    \frac{\tau_r}{\tau_f} = 0.848
\end{equation*}

Here we see very clearly that the positive and negative saturation values of \(I_{out}\) are different as discussed in section 2. 

\subsection{DC level manipulation}
We fixed the bias transistor voltage to 0.75 V and observed the circuit as we changed the DC level of the input from 2 V to 0.25 V in steps. Figs.~\ref{fig:ex4-2} through
\ref{fig:ex4-4} show the time-domain relationship between the input and output voltages measured at the transconductance amplifier terminals for several input bias voltages.
\begin{figure}[!htb]
    \center
    \includegraphics{ex4-1.eps}
    \caption{Input-output relationship in the time domain for a square wave input to the follower-integrator circuit with different input DC offsets.}
    \label{fig:ex4-2}
\end{figure}
\begin{figure}[!htb]
    \center
    \includegraphics{ex4-2.eps}
    \caption{Input-output relationship in the time domain for a square wave input to the follower-integrator circuit with different input DC offsets. Here the
    input transistors go into the ohmic region on the downgoing flank.}
    \label{fig:ex4-3}
\end{figure}
\begin{figure}[!htb]
    \center
    \includegraphics{ex4-3.eps}
    \caption{Input-output relationship in the time domain for a square wave input to the follower-integrator circuit for a very low DC offset which puts both
    input transistors in the ohmic region.}
    \label{fig:ex4-4}
\end{figure}
Initially, in Fig.~\ref{fig:ex4-2}, the output still faithfully follows the input because the input transistors are still operating in saturation. However the 
amplitude of the input and output signals go up because the input impedance changes.

In Fig.~\ref{fig:ex4-3} the input bias voltage is at a point where the differential input voltage holds one transistor in saturation and one in the triode region.
Resulting in the circuit being able to follow the upgoing edges, but not the downgoing ones because the current through the output transistors become very small
with the small drive.

Fig.~\ref{fig:ex4-4} shows the result of biasing both input transistors in the triode region and applying a signal that does not cause either of them to enter
saturation. The output current is now a triangle wave because of the constant current charging and discharging the capacitor.

\section{How Slow Can You Integrate?}
\begin{figure}
    \center
    \includegraphics{ex5-step.eps}
    \caption{Square wave input to the slow follower-integrator circuit and the output signal. The output signal is smoothed using a 50-sample delay adjusted moving
    average filter to clearly show the exponential voltage increase.}
    \label{fig:ex5-1}
\end{figure}
\begin{figure}
    \center
    \includegraphics{ex5-log.eps}
    \caption{The difference between the final and immediate values of the output voltage during a charge cycle with a line fit to find \(\tau = 1.7\)s.}
    \label{fig:ex5-2}
\end{figure}
Fig.~\ref{fig:ex5-1} shows the time-domain response of the transconductance amplifier follower-integrator circuit with a bias voltage of 0 V read from the multimeter
connected to the potbox, and an input bias of 2 V DC. The input is a 50 mV PP square wave. Fig.~\ref{fig:ex5-2} shows the difference between the final value of the
output and the immediate value. From this, the time constant of the circuit is found to be
\begin{equation*}
    \tau = 1.7 \mathrm{s}
\end{equation*}
When the follower-integrator circuit is made very slow, the offset moves from being positive to negative, relative to the applied square wave. It also
drops with the DC voltage.
\end{document}
